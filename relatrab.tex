\documentclass[12pt,a4paper]{article}
\usepackage[utf8]{inputenc}
\usepackage[portuguese]{babel}
\usepackage{amsmath}
\usepackage{amsfonts}
\usepackage{amssymb}
\usepackage{graphicx}
\usepackage{parskip}
\usepackage{setspace}
\usepackage{scrextend}
\usepackage{titling}
\usepackage{xcolor}
\usepackage{epigraph}
%\usepackage{draftwatermark}
%\SetWatermarkText{Em progresso}
%\SetWatermarkScale{3}

\newenvironment{citac}{\begin{addmargin}[4cm]{1em} \footnotesize}{\normalfont \end{addmargin}}

\usepackage[left=3.00cm, right=2.00cm, top=3.00cm, bottom=2.00cm]{geometry}
\author{Pedro T. R. Pinheiro}
\date{São Paulo\\2019}
\title{Relatório de Estágio}

\newcommand{\subtitulo}{O ensino de filosofia na Etec Raposo Tavares}
\newcommand{\disciplina}{EDF0424 - Metodologia do Ensino de Filosofia II}
\newcommand{\departamento}{Departamento de Metodologia do Ensino}
\newcommand{\unidade}{FE - Faculdade de Educação}
\newcommand{\prof}{Paulo H. F. Silveira}

\begin{document}
	\begin{center}
				\textbf{
				\LARGE USP - UNIVERSIDADE DE SÃO PAULO \\
			}
			\Large \textsc{\unidade} \\
			\large \textsc{\departamento}\\
			\vspace*{1cm}
				
			Disciplina: \disciplina; \\Prof.: \prof
			\vfill
			\begin{center}
				{\Large \textsc{\theauthor}} \\ 
				\vspace{1cm}
				\LARGE\textbf{\thetitle} \\
				\Large\emph{\subtitulo}
			\end{center}
			\vfill
			\large\thedate
			\vspace*{1cm}
			\thispagestyle{empty}			
	\end{center}

	\newpage

	\widowpenalty10000
	\clubpenalty10000
	\setlength{\parskip}{0.5cm}
	\setlength{\parindent}{1.1cm}
	\onehalfspacing
	
	\section{Preâmbulo}
	
	Este relatório trata dos eventos observados durante o estágio para a segunda 
	etapa da disciplina ``Metodologia de Ensino de Filosofia'', no segundo 
	semestre de 2019, em que foram acompanhadas as aulas de filosofia do 
	professor Délcio Nery, na Etec Raposo Tavares. O foco é fazer uma análise 
	da estrutura do curso, como os materiais e recursos disponíveis são 
	utilizados e, também, a resposta das turmas envolvidas às estratégias 
	correntemente adotadas. 
	
	Há de se frisar o imenso abismo entre as discussões realizadas no curso 
	e a dita ``realidade'': se por um lado temos, dentro dos muros da 
	universidade, a possibilidade de discutir métodos, ali, na prática, 
	defronte a uma sala de aula, prevalece o que coloquialmente poder-se-ia 
	rotular de ``vai como dá''. Como é de se esperar, a parte final deste 
	relatório procura, dentro da medida do possível, fechar este abismo. 
	
	Fechar este abismo, contudo, é meramente o primeiro passo. Com isto é 
	apenas possível diagnosticar algumas mazelas. É ainda maior desafio 
	propor soluções. Não se descortinam, ao jovem professor, de pronto, 
	a vasta gama de ferramentas para se trabalhar com os mais diversos 
	tipos de classes. Pelo contrário, são, inicialmente, muito limitados 
	os recursos com que conta o jovem licenciado. Deste modo, seria 
	demasiado ambicioso esperar deste relatório uma solução efetiva 
	às imperfeições observadas. 
	
	Logo, é apenas isto: uma tentativa. Tentativa de compreensão e 
	tentativa de solução. Afinal, sob certo ponto de vista, por toda a 
	volatilidade encontrada, e por toda a sua incerteza relacionada a 
	ser uma ciência que lida diretamente com os nervos da natureza 
	humana, não será ela inevitavelmente e quase exclusivamente empírica? 
	
	
	\newpage
	
	\section{Materiais e Organização do Curso}
	
    Délcio organiza suas aulas em pelo menos quatro modalidades diferentes: 
    temos aulas expositivas puras, exibição de filmes, debates livres e 
    atividades em grupo. As aulas expositivas são as mais comuns e geralmente 
    constituem numa espécie de ponta de lança para qualquer conteúdo. Quando 
    cabível, este é, geralmente, complementado por um filme, propriamente 
    editado para não durar mais do que umas duas aulas. Os debates livres 
    ocorrem no interlúdio e costumam ser conduzidos num ambiente externo, com 
    a intenção de aliviar a tensão dos alunos. Já a atividade em grupo é 
    realizada da metade para o fim do conteúdo. 
    
    Deste modo, do ponto de vista programático, se Délcio fosse, por exemplo, 
    discorrer sobre, digamos, ``forma e ideia na metafísica clássica'', 
    é seguro dizer que ao menos as duas primeiras aulas seriam dedicadas à 
    explicação de conceitos basilares (o que é forma, matéria, mimese etc.), 
    para depois avançar para algum tipo de filme ou material audiovisual 
    que trabalhe com o tema. Se houver tempo disponível, levará, numa quinta 
    aula, os alunos para o átrio, os fará sentar em forma de uma grande roda 
    e distribuirá a cada um deles um papel contendo uma pergunta. 
    
    Estas perguntas são bastante genéricas --- ``o que é alma para você?'', 
    ``o que é mais importante? Riqueza material ou riqueza intelectual'' e 
    assim por diante ---, tentando dialogar numa linguagem um pouco mais 
    próxima com a de um típico aluno do ensino básico. O aluno escolhido faz 
    a pergunta a alguém de sua escolha para que esta pessoa responda e repita 
    o procedimento. Com isto, Délcio pretende que os discentes consigam lidar 
    com as questões filosóficas de modo um pouco mais próximo do que 
    é tipicamente possível. 
    
    A atividade em grupo, que geralmente marca o encerramento de um conteúdo, 
    consiste numa folha frente e verso, contendo alguns trechos escritos por 
    filósofos ou seus comentadores e, em seguida, cerca de quatro ou cinco 
    questões, que, em boa parte, podem ser simplesmente respondidas com 
    uma simples interpretação de texto. Os alunos discutem entre si durante 
    a execução desta atividade, o que caracteriza este tipo de aula como algo 
    mais ``livre''. 
    
    Esta atividade, entretanto, tem entrega individual. Por vezes, começa no 
    fim da aula anterior ou excede o período da aula e torna-se tarefa de casa. 
    Segundo Délcio, esta atividade tem este caráter um pouco menos rígido 
    porque trata-se de uma forma ``leve'' de avaliação. Serve para ao menos 
    garantir que o discente adquira alguma assimilação individual dos temas 
    tratados em sala. 
    
    No tocante ao material didático empregado, Délcio não costuma usar livro 
    didático da escola. Seu material é frequentemente compilado de livros 
    ditáticos de sua coleção, tais como ``Convite à Filosofia'' de M. Chauí e 
    outros. O docente extrai alguns trechos 
    selecionados por estes livros e, eventualmente, algumas questões para uso 
    nas provas e/ou atividades. E estes excertos são justamente aqueles que 
    compõe as folhas da atividade, que são usadas para a atividade e 
    posteriormente devolvidas. 
    
    Como já fora dito um pouco acima, Délcio faz amplo uso de recursos 
    audiovisuais, sobretudo filmes. Em suas aulas, exibiu uma ampla seara 
    de títulos, tais como ``Black Mirror'', ``Vanilla Sky'', dentre outros. 
    São filmes que em geral englobam, por excelência, questões filosóficas 
    consagradas. Em Vanilla Sky, por exemplo, discute-se o tema da chamada 
    \textit{criogenia}, uma das múltiplas tentativas do homem de vencer a 
    mortalidade. 
    
    Para Délcio, filmes são um poderoso meio de transmissão de conteúdos 
    filosóficos, pois não tratam dos temas filosóficos de uma forma técnica e 
    ``direta ao ponto''. Dissolvem-no numa história e, ainda, de quebra, dão ao 
    aluno o desafio de depreender a mensagem do enredo. Idealmente, a sala tem, 
    ao fim de cada exibição, cerca de dez a vinte minutos para discutir acerca 
    do filme dentro do contexto do tema em discussão no momento do curso. 
    
    A resposta ao conteúdo varia de sala para sala, como é, aliás, de se 
    esperar. Um dos maiores problemas aqui --- e, em geral, bastante típico 
    dos oferecimentos de filosofia --- é a falta de tempo: gasta-se 
    uma considerável fatia deste organizando-se a sala, da chamada à disposição 
    dos tópicos no quadro negro. Além disso, o \textit{attention span} dos 
    alunos é, em algumas turma, não muito superior à marca dos quinze minutos. 
    Isto, somado ao fato de haver apenas uma aula semanal (ou duas, no caso de 
    algumas salas --- 1º DS e 1º AQI, em nosso caso), faz com que, novamente, 
    observe-se uma aproximação bastante ``horizontal'' ao conteúdo. 
    	
	\newpage
	
	\section{As aulas}

    \subsection*{Aula I}
    
    Ministrada no dia 12/09, fora um debate sobre filosofia da ciência. Délcio 
    chama os alunos para se sentarem numa área livre do pátio e lhes distribui 
    papéis com questões relativas ao conteúdo. Entre chamada e organização da 
    roda transcorrem onze minutos. Quando estão, todos, finalmente, com suas 
    perguntas em mãos e organizados em roda, Délcio faz sinal para 
    Samara\footnote{Todos os nomes de alunos aqui citados são fictícios} 
    inicie. 
    
    Ela começa lendo a pergunta, interpelada pelo burburinho de alguns 
    outros alunos. Délcio e mais duas alunas pedem silêncio aos colegas, e 
    Samara prossegue, indagando a Cássio: ``Na sua concepção, por que existem 
    pessoas que defendem a teoria da terra plana?'' Cássio responde citando 
    que é resultado de falta de educação e que nem todas as pessoas tem acesso 
    adequado às informações. Alguns alunos adentram a discussão e sugerem que 
    este comportamento só pode mesmo ser algum tipo de ``zoeira'', pois não 
    seria possível pensar assim, dado um amplo conjunto de informações 
    disponíveis que corrobore o fato de a terra ser redonda. 
    
    Quando retorna a discussão a Cássio, ele arremata dizendo acreditar que 
    existem, sim, pessoas que defendem esta concepção porque ``o mapa é plano 
    e as pessoas difundem qualquer crendice''. Cássio, em seguida, faz a 
    Arthur a pergunta que está em suas mãos ``O Brasil é um país que 
    investe pouco em ciência?''. Ao que Arthur replica: ``Mas é lógico, 
    se investisse (sic) mais, a gente (sic) sipá (sic) estaria lá com os 
    países europeus em medicina e essas coisas...'' 
    
    Neste ponto, a sala concordou, mas alguns sugerem que a ciência deve 
    avançar com cautela. Pois, do contrário, pode representar uma espécie de 
    repressão determista. 
    
    Agora que já temos uma ideia do teor da discussão basta dizer que seguiu 
    mais ou menos estes moldes até o fim do encontro. Houve algumas 
    interruptações, no meio percurso, mas a sala se portou de forma 
    relativamente madura. 

    \subsection*{Aula II}
    
    O segundo encontro aqui relatado, foi no mesmo dia, com a sala do primeiro 
    ano do ensino médio integrado ao curso técnico de desenvolvimento de 
    sistemas. Trata-se de uma aula argumentativa feita nos mesmos moldes da 
    anterior. O tema, contudo, era, aqui, outro: \textit{criogenia} e 
    \textit{eutanásia}, no contexto da discussão acerca da ciência 
    contemporânea. 
    
    Novamente, a sala demora cerca de dez minutos para estar organizada, 
    numa sombra do pátio externo, em forma de roda. Délcio distribuiu as 
    folhinhas contendo as questões e quem inicia o debate é Tamires. Ela indaga 
    a Clayton, um dos alunos mais tímidos da sala, o que está contido em seu 
    papel --- ``Se você tivesse a opção de se congelar para depois, no futuro,  
    tornar-se imortal, você o faria?''. 
    
    Ao que Clayton tenta, entre interrupções de seus colegas, que o caçoam por 
    ser excessivamente acanhado, responder que sim. Mas fica envergonhado em 
    justificar sua resposta. Após alguma insistência por parte de Délcio, 
    o rapaz diz algo que, a bem da verdade, foi ininteligível. Para não quebrar 
    o fluxo da aula, resolveu-se escolher outro aluno. Apontaram Lorena, que 
    indaga a Raphael ``Se você tivesse uma doença crônica, ainda assim 
    você optaria pela criogenia, sabendo que há pouca chance de cura, mesmo 
    no futuro?''. Raphael fica pensativo. Após alguns segundos dispara que 
    ``sim, pois preferia arriscar essa chance do que ser simplesmente 
    esquecido para todo o sempre.'' 
    
    Nisto, Carlos, entra na discussão dizendo que preferia a eutanásia para 
    estes casos. Délcio, estupefato, afirma que uma coisa não exclui a outra, 
    e que alguém pode sim optar por passar por eutanásia e depois ser congelado. 
    Muitos alunos ficam perplexos. Carlos procura se justificar citando que 
    não quer ter qualquer tipo de sobrevida após sua morte: quer que sua 
    família passe pelo que considera ser um natural processo de luto e que, 
    além disso, o processo criogênico é bastante caro, inacessível a pessoas 
    com poucos recursos. 
    
    Délcio elogia a resposta de Carlos e pontua a relevância do processo de 
    luto para o ser humano, dando aos familiares a chance, talvez, de 
    ``sentir a perda e não a falta.'' Pede, então, para Carlos escolha o próximo 
    aluno  a fazer uma pergunta. Em seguida, continuaram até o fim do 
    encontro no tema da eutanásia. Alguns alunos comentaram ter certo ``medo'' 
    da morte e que a criogenia lhes parecia uma saída viável para, talvez, 
    ``contarem o que viram''. Fora isso, as intervenções foram mais ou menos 
    todas no mesmo grau já observado. 
    
    \subsection*{Aula III}
    
    Terceiro encontro aqui relatado: dia 26/09, uma aula expositiva sobre 
    filosofia da ciência e Aristóteles, ministrada para o 1º AQI (primeiro ano
    do ensino médio integrado ao curso de técnico de química). Délcio faz a 
    chamada e em pouco mais de cinco minutos, inicia a aula, depois de dispor 
    no quadro negro, os pontos chave do conteúdo. 
    
    É uma aula clássica, conteudista. Délcio cita algumas passagens do livro 
    Física, Meterologia e do Organon e parte delas para explicar a busca na 
    filosofia pela lógica. Alguns alunos fazem expressão de dúvida, mas uma boa 
    parte da sala está atenta. Uns poucos discentes, ao fundo da sala, 
    optam francamente por não prestar atenção, mas não atrapalham. A aluna 
    Mariana foi a única a levantar a mão, ao fim da explicação e disse não 
    compreender bem a ideia de ``materia''. Délcio optou por explicar a ela 
    após o término da aula. 
    
    Findada a explicação, que levou cerca de dezessete minutos, Délcio 
    distribuiu uma folha contendo alguns excertos de textos clássicos de 
    Aristóteles e algumas perguntas de interpretação de texto. A ideia é que 
    os alunos começassem esta atividade e, caso não terminasse, a entregassem 
    no próximo encontro. 
    
    \subsection*{Aula IV}

	Quinta-feira, dia dez de outubro. Era a primeira aula da manhã no dito 3º 
	AII (terceiro ano, integrado ao técnico em informática). Estes futuros 
	informáticos não se encontravam em seu melhor dia: pelo contrário, se 
	demoraram a organizarsem para a exposição. Délcio pretendia usar esta aula 
	para arrematar o tópico de filosofia política, pois pretendia passar, no 
	próximo encontro, uma atividade acerca deste assunto, que, enfim, o 
	encerraria. 
	
	Após dez minutos de uma intensa desordem, Délcio faz a chamada, respondida 
	aos berros pelos alunos que conseguiam escutar seus nomes através do 
	poluído espectro acústico. Levanta-se e, procurando ignorar a pândega que 
	ocorria ao fundo, pôs-se a escrever os tópicos dos quais trataria na 
	exposição. Dispunham-se ao quadro os nomes de T. Hobbes, J. Rousseau e, 
	para arrematar, J. Locke. Sob cada um deles, estavam listados os aspectos 
	mais importantes e destacados das principais obras. Pontos que seriam 
	embasados por recortes presentes na folha da atividade, distribuída ao 
	fim deste encontro. 
	
	Délcio pede silêncio. Um breve silêncio dá ao docente a chance de iniciar 
	a exposição. A cada dois ou três minutos é preciso chamar a atenção da 
	sala para manter o volume da conversa baixo o suficiente para conseguir 
	falar com os poucos discentes interessados no tema. Em pouco mais do que 
	quinze minutos, Délcio expôs os nomes das obras principais de casa 
	filósofo citado e tentou emendar o tema do contratualismo com o sistema 
	de governo e leis presente, citando Montesquieu. Fora, entretanto de 
	relance, a menção. E, dentre o ruído de fundo, praticamente 
	passara batida. 
	
	Antes que Délcio findasse a explicação, interpela-me um aluno, indangando 
	jocosamente sobre qual a principal atividade de um filósofo, ou, ipsis 
	literis ``e aí, estagiário, depois que cês terminam a faculdade cês vendem 
	arte na praia ou vira mendigo da Paulista''. Retorqui em tom igualmente 
	jocoso e tornei a observar o encontro. 
	
	Délcio distribui os papéis de atividade. Faltando cerca de quinze minutos 
	para o fim da aula, é esperado que os alunos façam perguntas e depois 
	comecem a fazer a atividade, cuja continuação é prevista para o próximo 
	encontro. 
	
	\subsection*{Aula V}
	
	Numa terça-feira, Délcio, pretendia passar, ainda no 1º DS, uma atividade 
	sobre filosofia da ciência e Aristóteles, como tratado no encontro 
	anterior. A turma, então relatou ter tido certa dificuldade com o conteúdo, 
	e muitos ainda estavam entregando a atividade começada na aula anterior. 
	Os alunos pediram uma pequena revisão deste conteúdo e, também, dos 
	anteriores.  
	
	Apesar de ser um docente bastante respeitado na Etec, neste encontro foi 
	pego no contrapé, pois não conseguira, a contento explicar o tópico à 
	turma. Contrariado, fez da atividade um trabalho para casa e fez a chamada. 
	Nisto, restavam cerca de vinte e cinco minutos para o fim do encontro, 
	tempo que algumas alunas usaram para conversar sobre um assunto 
	anteriormente discutido que seria tema da avaliação, dali a alguns dias. 
	
	Outros discentes fizeram o mesmo, algum tempo depois. Ainda houve uns 
	poucos alunos que acharam por bem entregar a atividade ainda na aula. 
	Faltando cerca de sete minutos para o final, Délcio anuncia os temas 
	da avaliação novamente (já o havia feito num encontro anterior), e, 
	após responder duas ou três perguntas, despede-se da turma e retira-se. 

	\subsection*{Aula VI}
	
	Dia sete de novembro, uma quinta-feira. Segunda aula de um dia bastante 
	quente. Délcio chamou os alunos para sentarem-se no pátio. Era novamente 
	o 1º DS. Esta sala é, para os padrões da Etec, uma classe bastante 
	comportada, atenta, e, sobretudo, dócil. 
	A despeito disso, este tipo de aula sempre demanda 
	um deslocamento e um certo tempo para que os alunos se organizem e 
	encontrem seus lugares. Isto leva de dez a quinze minutos. Neste dia 
	levou pouco mais de dez, fora os cinco minutos iniciais usados para 
	o docente dispor seus pertences e fazer a chamada. 
	
	Enfim dispostos, os alunos recebem os papéis de Délcio. Era mais uma 
	das famosas rodas de discussão do professor. E, desta vez, o tema era 
	ciência, com foco em tecnologia. Tainá fora a primeira aluna escolhida, 
	e a pergunta que estava em suas mãos era “Em sua opinião, a tecnologia 
	afasta ou aproxima as pessoas?”, direcionada a Murilo, que a retorque, 
	encabuladamente, que “sipá (sic) os dois”. Délcio pede mais detalhes. 
	Murilo esboça que talvez ``a gente se aproxime por mais tempo, mas 
	para de prestar atenção no que tá acontecendo em volta''. Tamires e 
	Carlos tecem alguns comentários, em meio a uma discussão. Dizem que 
	antes ``não tinha como'' conversar com amigos e parentes do outro 
	lado do oceano sem ``pagar mó grana'' e que haviam ferramentas como 
	Skype, WhatsApp etc. 
	
	Délcio argumenta que, embora seja evidente o potencial da tecnologia 
	em unir pessoas para além das barreiras geográficas, pode haver nela 
	um caráter secundário que talvez não estivesse sendo observado pelos 
	dois alunos. ``Tudo ficou fácil, próximo e rápido (...) será que tudo 
	não acaba sendo tão rápido que nem acaba dando tempo para pensar?''
	E, então, aponta para Adamastor. 
	
	Adamastor, cuja pergunta é ``É possível se sentir só com tantas redes 
	sociais e tantas pessoas para conversar a um clique de distância?''. 
	O aluno dirige a pergunta a Nathália, que, aparente campeã no aplicativo 
	``Instagram'', responde ``em geral não, mas é uma coisa meio vazia, 
	sabe? Todo mundo vai lá curte, mas cê não sabe se a pessoa só te 
	aprecia pelo corpo e não pelo que você é de verdade''. A esta altura, 
	Perseu, um aluno um pouco mais retraído, faz um comentário e afirma 
	que as redes parecem só aumentar a solidão, que não ser notado nem 
	na rede e nem na vida real faz com que a pessoa se diminua. Lorena 
	complementa que a felicidade do ``like'' dura pouco e é tão viciante, 
	que quando você não recebe, sofre de abstinência, ``como se fosse 
	uma droga''. Délcio parabeniza a aluna pelo comentário. 
	
	Raphael emenda uma breve história de um acontecimento pessoal envolvendo 
	a mesma rede social anteriormente citada e o Twitter. Aqui a se omite, 
	por não ser de significativa relevância para o relato da aula em si. 
	Ele, então, é escolhido a fazer uma pergunta e escolhe Lorena para 
	responder a seguinte questão: ``O ser humano deve depender da tecnologia 
	ou deve procurar o máximo possível nunca perder de vista a possibilidade 
	de ter de viver sem ela?''. 
	
	Para Lorena, a resposta é que precisamos ``nos conectar com a natureza'' 
	e ``tentar entender que a vida é mais do que celular, internet e 
	videogame''. Diz ser tudo isto bastante interessante e instigante, mas 
	depender disso é ``como depender das drogas''. Nisto, Délcio a indaga se 
	a medicina deveria voltar a usar raízes e instrumentos rudimentares para 
	tratar os doentes. Intrigada, Lorena concede já ser o homem dependente 
	de uma série de tecnologias, mas que é preciso saber que nem sempre dá 
	para contar com elas, ``tipo na África lá, as pessoas não tem nem energia 
	direito''. Raphael aponta que ``tecnologia é importante, mas que gente 
	precisa ter noção de pode acontecer um apocalipse zumbi ou sei lá e a 
	gente vai estar tudo na rua (sic) e metade do que a gente acha que tá 
	sempre 	aí pode não tar (sic), né?''
	
	Clayton afirma que ``o futuro é a tecnologia'', argumentando que o homem
	só se desenvolveu porque criou ``tecnologia em cima de tecnologia''. 
	Carlos e Sabrina concordam e tecem mais alguns comentários, até Délcio
	interrompê-los, dando chance a Milena de repassar a última questão --- 
	``A escola como conhecemos vai sempre existir a despeito da tecnologia 
	ou as relações e formas de transmissão de conteúdo se modificarão 
	drasticamente no futuro?'' --- a Aristides, que a redargue que sim. 
	Aristides, um rapaz de poucas palavras, diz que a escola ``mudou mó (sic)
	pouco'', mesmo ``a internet tando (sic) aí há maior quota''. Milena a 
	indaga que se é mesmo a mesma coisa ir à escola há mais de vinte anos 
	atrás ``sem Google nem nada'' e hoje, ``com o 3G e o Google na ponta 
	do dedo, aí''. 
	
	Aristides reconhece que não. Mas limita-se a dizer que só não acredita 
	que as coisas mudarão. A discussão, entretanto, não engata. A discussão 
	já avançava dois minutos além do horário previsto. Délcio libera os 
	alunos para a sala de aula, mas não sem lembrá-los de que ao menos cinco 
	pessoas deixaram de entregar-lhe um dos trabalhos pedidos duas aulas antes. 
	
	
	\newpage
		
	\section{O curso em vista dos parâmetros nacionais}
	
    
	
	\newpage
	
	\section{O curso ante a bibliografia desta disciplina}
	

	
	
	\newpage
	
	\section{Epílogo}
	
    
	
	\newpage
	

	
	\newpage
	
	\begin{thebibliography}{9}
		\bibitem{adorno}
		ADORNO, T. 
		\textit{Education after Auschwitz. }   
		\\\texttt{https://www.ime.usp.br/~tadeu/EDF0285/A10\_Adorno.pdf}
		
		
	\end{thebibliography}

\end{document}

