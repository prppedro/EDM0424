\documentclass[12pt,a4paper]{article}
\usepackage[utf8]{inputenc}
\usepackage[portuguese]{babel}
\usepackage{amsmath}
\usepackage{amsfonts}
\usepackage{amssymb}
\usepackage{graphicx}
\usepackage{parskip}
\usepackage{setspace}
\usepackage{scrextend}
\usepackage{titling}
\usepackage{xcolor}
\usepackage{epigraph}
%\usepackage{draftwatermark}
%\SetWatermarkText{Em progresso}
%\SetWatermarkScale{3}

\newenvironment{citac}{\begin{addmargin}[4cm]{1em} \footnotesize}{\normalfont \end{addmargin}}

\usepackage[left=3.00cm, right=2.00cm, top=3.00cm, bottom=2.00cm]{geometry}
\author{Pedro T. R. Pinheiro}
\date{São Paulo\\2019}
\title{Relatório de Estágio}

\newcommand{\subtitulo}{O ensino de filosofia na Etec Raposo Tavares}
\newcommand{\disciplina}{EDF0424 - Metodologia do Ensino de Filosofia II}
\newcommand{\departamento}{Departamento de Metodologia do Ensino}
\newcommand{\unidade}{FE - Faculdade de Educação}
\newcommand{\prof}{Paulo H. F. Silveira}

\begin{document}
	\begin{center}
				\textbf{
				\LARGE USP - UNIVERSIDADE DE SÃO PAULO \\
			}
			\Large \textsc{\unidade} \\
			\large \textsc{\departamento}\\
			\vspace*{1cm}
				
			Disciplina: \disciplina; \\Prof.: \prof
			\vfill
			\begin{center}
				{\Large \textsc{\theauthor}} \\ 
				\vspace{1cm}
				\LARGE\textbf{\thetitle} \\
				\Large\emph{\subtitulo}
			\end{center}
			\vfill
			\large\thedate
			\vspace*{1cm}
			\thispagestyle{empty}			
	\end{center}

	\newpage

	\widowpenalty10000
	\clubpenalty10000
	\setlength{\parskip}{0.5cm}
	\setlength{\parindent}{1.1cm}
	\onehalfspacing
	
	\section{Preâmbulo}
	
    É de particular importância à formação do professor de filosofia observar 
    o ambiente de trabalho em que está inserido. Afinal, não pódem lhe 
    perpassar as particularidades sociais e conjunturais enquanto instrui seus 
    alunos. Esta observação, se atenta o suficiente, pode evitar que o docente 
    dê “murros em ponta de faca” ou mesmo adote vias de trabalho que não 
    funcionam. É neste ponto em que são postas a prova as teorias aprendidas 
    em didática e psicologia da educação. Portanto, um relatório de estágio 
    constitui num dos mais importantes exercícios interpretativos para um 
    docente em formação. 

    Neste trabalho, acompanha-se o docente Délcio Nery, da Etec Raposo Tavares, 
    já acompanhado anteriormente por este que vos escreve, no módulo anterior 
    da disciplina de Metodologia de Ensino de Filosofia. Neste semestre, o foco 
    fora analisar o curso de filosofia, tal como organizado por Délcio. Que 
    estratégias adota, que materiais utiliza etc. dentro dos parâmetros 
    fornecidos para a feitura deste relatório. 
	
	
	\newpage
	
	\section{Materiais e Organização do Curso}
	
    Délcio organiza suas aulas em pelo menos quatro modalidades diferentes: 
    temos aulas expositivas puras, exibição de filmes, debates livres e 
    atividades em grupo. As aulas expositivas são as mais comuns e geralmente 
    constituem numa espécie de ponta de lança para qualquer conteúdo. Quando 
    cabível, este é, geralmente, complementado por um filme, propriamente 
    editado para não durar mais do que umas duas aulas. Os debates livres 
    ocorrem no interlúdio e costumam ser conduzidos num ambiente externo, com 
    a intenção de aliviar a tensão dos alunos. Já a atividade em grupo é 
    realizada da metade para o fim do conteúdo. 
    
    Deste modo, do ponto de vista programático, se Délcio fosse, por exemplo, 
    discorrer sobre, digamos, ``forma e ideia na metafísica clássica'', 
    é seguro dizer que ao menos as duas primeiras aulas seriam dedicadas à 
    explicação de conceitos basilares (o que é forma, matéria, mimese etc.), 
    para depois avançar para algum tipo de filme ou material audiovisual 
    que trabalhe com o tema. Se houver tempo disponível, levará, numa quinta 
    aula, os alunos para o átrio, os fará sentar em forma de uma grande roda 
    e distribuirá a cada um deles um papel contendo uma pergunta. 
    
    Estas perguntas são bastante genéricas --- ``o que é alma para você?'', 
    ``o que é mais importante? Riqueza material ou riqueza intelectual'' e 
    assim por diante ---, tentando dialogar numa linguagem um pouco mais 
    próxima com a de um típico aluno do ensino básico. O aluno escolhido faz 
    a pergunta a alguém de sua escolha para que esta pessoa responda e repita 
    o procedimento. Com isto, Délcio pretende que os discentes consigam lidar 
    com as questões filosóficas de modo um pouco mais próximo do que 
    é tipicamente possível. 
    
    A atividade em grupo, que geralmente marca o encerramento de um conteúdo, 
    consiste numa folha frente e verso, contendo alguns trechos escritos por 
    filósofos ou seus comentadores e, em seguida, cerca de quatro ou cinco 
    questões, que, em boa parte, podem ser simplesmente respondidas com 
    uma simples interpretação de texto. Os alunos discutem entre si durante 
    a execução desta atividade, o que caracteriza este tipo de aula como algo 
    mais ``livre''. 
    
    Esta atividade, entretanto, tem entrega individual. Por vezes, começa no 
    fim da aula anterior ou excede o período da aula e torna-se tarefa de casa. 
    Segundo Délcio, esta atividade tem este caráter um pouco menos rígido 
    porque trata-se de uma forma ``leve'' de avaliação. Serve para ao menos 
    garantir que o discente adquira alguma assimilação individual dos temas 
    tratados em sala. 
    
    No tocante ao material didático empregado, Délcio não costuma usar livro 
    didático da escola. Seu material é frequentemente compilado de livros 
    ditáticos de sua coleção, tais como ``Convite à Filosofia'' de M. Chauí e 
    \textbf{ARRUMAR OUTRO PARA PÔR AQUI}. O docente extrai alguns trechos 
    selecionados por estes livros e, eventualmente, algumas questões para uso 
    nas provas e/ou atividades. E estes excertos são justamente aqueles que 
    compõe as folhas da atividade, que são usadas para a atividade e 
    posteriormente devolvidas. 
    
    Como já fora dito um pouco acima, Délcio faz amplo uso de recursos 
    audiovisuais, sobretudo filmes. Em suas aulas, exibiu uma ampla seara 
    de títulos, tais como ``Black Mirror'', ``Vanilla Sky'', dentre outros. 
    São filmes que em geral englobam, por excelência, questões filosóficas 
    consagradas. Em Vanilla Sky, por exemplo, discute-se o tema da chamada 
    \textit{criogenia}, uma das múltiplas tentativas do homem de vencer a 
    mortalidade. 
    
    Para Délcio, filmes são um poderoso meio de transmissão de conteúdos 
    filosóficos, pois não tratam dos temas filosóficos de uma forma técnica e 
    ``direta ao ponto''. Dissolvem-no numa história e, ainda, de quebra, dão ao 
    aluno o desafio de depreender a mensagem do enredo. Idealmente, a sala tem, 
    ao fim de cada exibição, cerca de dez a vinte minutos para discutir acerca 
    do filme dentro do contexto do tema em discussão no momento do curso. 
    
    A resposta ao conteúdo varia de sala para sala, como é, aliás, de se 
    esperar. Um dos maiores problemas aqui --- e, em geral, bastante típico 
    dos oferecimentos de filosofia --- é a falta de tempo: gasta-se 
    uma considerável fatia deste organizando-se a sala, da chamada à disposição 
    dos tópicos no quadro negro. Além disso, o \textit{attention span} dos 
    alunos é, em algumas turma, não muito superior à marca dos quinze minutos. 
    Isto, somado ao fato de haver apenas uma aula semanal (ou duas, no caso de 
    algumas salas --- 1º DS e 1º AQI, em nosso caso), faz com que, novamente, 
    observe-se uma aproximação bastante ``horizontal'' ao conteúdo. 
    
    
    	
	\newpage
	
	\section{As aulas}

    \subsection*{Aula I}

    \subsection*{Aula II}

    \subsection*{Aula III}
    
    \subsection*{Aula IV}

    \subsection*{Aula V}

    \subsection*{Aula VI}
	
	\newpage
		
	\section{O curso em vista dos parâmetros nacionais}
	

	

	
	\newpage
	
	\section{O curso ante a bibliografia desta disciplina}
	

	
	
	\newpage
	
	\section{Epílogo}
	

	
	\newpage
	

	
	\newpage
	
	\begin{thebibliography}{9}
		\bibitem{adorno}
		ADORNO, T. 
		\textit{Education after Auschwitz. }   
		\\\texttt{https://www.ime.usp.br/~tadeu/EDF0285/A10\_Adorno.pdf}
		
		
	\end{thebibliography}

\end{document}

