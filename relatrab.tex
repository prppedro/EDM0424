\documentclass[12pt,a4paper]{article}
\usepackage[utf8]{inputenc}
\usepackage[portuguese]{babel}
\usepackage{amsmath}
\usepackage{amsfonts}
\usepackage{amssymb}
\usepackage{graphicx}
\usepackage{parskip}
\usepackage{setspace}
\usepackage{scrextend}
\usepackage{titling}
\usepackage{xcolor}
\usepackage{epigraph}
%\usepackage{draftwatermark}
%\SetWatermarkText{Em progresso}
%\SetWatermarkScale{3}

\newenvironment{citac}{\begin{addmargin}[4cm]{1em} \footnotesize}{\normalfont \end{addmargin}}

\usepackage[left=3.00cm, right=2.00cm, top=3.00cm, bottom=2.00cm]{geometry}
\author{Pedro T. R. Pinheiro}
\date{São Paulo\\2019}
\title{Relatório de Estágio}

\newcommand{\subtitulo}{O ensino de filosofia na Etec Raposo Tavares}
\newcommand{\disciplina}{EDF0424 - Metodologia do Ensino de Filosofia II}
\newcommand{\departamento}{Departamento de Metodologia do Ensino}
\newcommand{\unidade}{FE - Faculdade de Educação}
\newcommand{\prof}{Paulo H. F. Silveira}

\begin{document}
	\begin{center}
				\textbf{
				\LARGE USP - UNIVERSIDADE DE SÃO PAULO \\
			}
			\Large \textsc{\unidade} \\
			\large \textsc{\departamento}\\
			\vspace*{1cm}
				
			Disciplina: \disciplina; \\Prof.: \prof
			\vfill
			\begin{center}
				{\Large \textsc{\theauthor}} \\ 
				\vspace{1cm}
				\LARGE\textbf{\thetitle} \\
				\Large\emph{\subtitulo}
			\end{center}
			\vfill
			\large\thedate
			\vspace*{1cm}
			\thispagestyle{empty}			
	\end{center}

	\newpage

	\widowpenalty10000
	\clubpenalty10000
	\setlength{\parskip}{0.5cm}
	\setlength{\parindent}{1.1cm}
	\onehalfspacing
	
	\section{Preâmbulo}
	
	Este relatório trata dos eventos observados durante o estágio para a segunda 
	etapa da disciplina ``Metodologia de Ensino de Filosofia'', no segundo 
	semestre de 2019, em que foram acompanhadas as aulas de filosofia do 
	professor Délcio Nery, na Etec Raposo Tavares. O foco é fazer uma análise 
	da estrutura do curso, como os materiais e recursos disponíveis são 
	utilizados e, também, a resposta das turmas envolvidas às estratégias 
	correntemente adotadas. 
	
	
	
	
	\newpage
	
	\section{Materiais e Organização do Curso}
	
    Délcio organiza suas aulas em pelo menos quatro modalidades diferentes: 
    temos aulas expositivas puras, exibição de filmes, debates livres e 
    atividades em grupo. As aulas expositivas são as mais comuns e geralmente 
    constituem numa espécie de ponta de lança para qualquer conteúdo. Quando 
    cabível, este é, geralmente, complementado por um filme, propriamente 
    editado para não durar mais do que umas duas aulas. Os debates livres 
    ocorrem no interlúdio e costumam ser conduzidos num ambiente externo, com 
    a intenção de aliviar a tensão dos alunos. Já a atividade em grupo é 
    realizada da metade para o fim do conteúdo. 
    
    Deste modo, do ponto de vista programático, se Délcio fosse, por exemplo, 
    discorrer sobre, digamos, ``forma e ideia na metafísica clássica'', 
    é seguro dizer que ao menos as duas primeiras aulas seriam dedicadas à 
    explicação de conceitos basilares (o que é forma, matéria, mimese etc.), 
    para depois avançar para algum tipo de filme ou material audiovisual 
    que trabalhe com o tema. Se houver tempo disponível, levará, numa quinta 
    aula, os alunos para o átrio, os fará sentar em forma de uma grande roda 
    e distribuirá a cada um deles um papel contendo uma pergunta. 
    
    Estas perguntas são bastante genéricas --- ``o que é alma para você?'', 
    ``o que é mais importante? Riqueza material ou riqueza intelectual'' e 
    assim por diante ---, tentando dialogar numa linguagem um pouco mais 
    próxima com a de um típico aluno do ensino básico. O aluno escolhido faz 
    a pergunta a alguém de sua escolha para que esta pessoa responda e repita 
    o procedimento. Com isto, Délcio pretende que os discentes consigam lidar 
    com as questões filosóficas de modo um pouco mais próximo do que 
    é tipicamente possível. 
    
    A atividade em grupo, que geralmente marca o encerramento de um conteúdo, 
    consiste numa folha frente e verso, contendo alguns trechos escritos por 
    filósofos ou seus comentadores e, em seguida, cerca de quatro ou cinco 
    questões, que, em boa parte, podem ser simplesmente respondidas com 
    uma simples interpretação de texto. Os alunos discutem entre si durante 
    a execução desta atividade, o que caracteriza este tipo de aula como algo 
    mais ``livre''. 
    
    Esta atividade, entretanto, tem entrega individual. Por vezes, começa no 
    fim da aula anterior ou excede o período da aula e torna-se tarefa de casa. 
    Segundo Délcio, esta atividade tem este caráter um pouco menos rígido 
    porque trata-se de uma forma ``leve'' de avaliação. Serve para ao menos 
    garantir que o discente adquira alguma assimilação individual dos temas 
    tratados em sala. 
    
    No tocante ao material didático empregado, Délcio não costuma usar livro 
    didático da escola. Seu material é frequentemente compilado de livros 
    ditáticos de sua coleção, tais como ``Convite à Filosofia'' de M. Chauí e 
    \textbf{Error: missing reference}. O docente extrai alguns trechos 
    selecionados por estes livros e, eventualmente, algumas questões para uso 
    nas provas e/ou atividades. E estes excertos são justamente aqueles que 
    compõe as folhas da atividade, que são usadas para a atividade e 
    posteriormente devolvidas. 
    
    Como já fora dito um pouco acima, Délcio faz amplo uso de recursos 
    audiovisuais, sobretudo filmes. Em suas aulas, exibiu uma ampla seara 
    de títulos, tais como ``Black Mirror'', ``Vanilla Sky'', dentre outros. 
    São filmes que em geral englobam, por excelência, questões filosóficas 
    consagradas. Em Vanilla Sky, por exemplo, discute-se o tema da chamada 
    \textit{criogenia}, uma das múltiplas tentativas do homem de vencer a 
    mortalidade. 
    
    Para Délcio, filmes são um poderoso meio de transmissão de conteúdos 
    filosóficos, pois não tratam dos temas filosóficos de uma forma técnica e 
    ``direta ao ponto''. Dissolvem-no numa história e, ainda, de quebra, dão ao 
    aluno o desafio de depreender a mensagem do enredo. Idealmente, a sala tem, 
    ao fim de cada exibição, cerca de dez a vinte minutos para discutir acerca 
    do filme dentro do contexto do tema em discussão no momento do curso. 
    
    A resposta ao conteúdo varia de sala para sala, como é, aliás, de se 
    esperar. Um dos maiores problemas aqui --- e, em geral, bastante típico 
    dos oferecimentos de filosofia --- é a falta de tempo: gasta-se 
    uma considerável fatia deste organizando-se a sala, da chamada à disposição 
    dos tópicos no quadro negro. Além disso, o \textit{attention span} dos 
    alunos é, em algumas turma, não muito superior à marca dos quinze minutos. 
    Isto, somado ao fato de haver apenas uma aula semanal (ou duas, no caso de 
    algumas salas --- 1º DS e 1º AQI, em nosso caso), faz com que, novamente, 
    observe-se uma aproximação bastante ``horizontal'' ao conteúdo. 
    	
	\newpage
	
	\section{As aulas}
	
	[fill this]

    \subsection*{Aula I}
    
    Ministrada no dia 12/09, fora um debate sobre filosofia da ciência. Délcio 
    chama os alunos para se sentarem numa área livre do pátio e lhes distribui 
    papéis com questões relativas ao conteúdo. Entre chamada e organização da 
    roda transcorrem onze minutos. Quando estão, todos, finalmente, com suas 
    perguntas em mãos e organizados em roda, Délcio faz sinal para 
    Samara\footnote{Todos os nomes de alunos aqui citados são fictícios} 
    inicie. 
    
    Ela começa lendo a pergunta, interpelada pelo burburinho de alguns 
    outros alunos. Délcio e mais duas alunas pedem silêncio aos colegas, e 
    Samara prossegue, indagando a Cássio: ``Na sua concepção, por que existem 
    pessoas que defendem a teoria da terra plana?'' Cássio responde citando 
    que é resultado de falta de educação e que nem todas as pessoas tem acesso 
    adequado às informações. Alguns alunos adentram a discussão e sugerem que 
    este comportamento só pode mesmo ser algum tipo de ``zoeira'', pois não 
    seria possível pensar assim, dado um amplo conjunto de informações 
    disponíveis que corrobore o fato de a terra ser redonda. 
    
    Quando retorna a discussão a Cássio, ele arremata dizendo acreditar que 
    existem, sim, pessoas que defendem esta concepção porque ``o mapa é plano 
    e as pessoas difundem qualquer crendice''. Cássio, em seguida, faz a 
    Arthur a pergunta que está em suas mãos ``O Brasil é um país que 
    investe pouco em ciência?''. Ao que Arthur replica: ``Mas é lógico, 
    se investisse (sic) mais, a gente (sic) sipá (sic) estaria lá com os 
    países europeus em medicina e essas coisas...'' 
    
    Neste ponto, a sala concordou, mas alguns sugerem que a ciência deve 
    avançar com cautela. Pois, do contrário, pode representar uma espécie de 
    repressão determista. 
    
    Agora que já temos uma ideia do teor da discussão basta dizer que seguiu 
    mais ou menos estes moldes até o fim do encontro. Houve algumas 
    interruptações, no meio percurso, mas a sala se portou de forma 
    relativamente madura. 

    \subsection*{Aula II}
    
    O segundo encontro aqui relatado, foi no mesmo dia, com a sala do primeiro 
    ano do ensino médio integrado ao curso técnico de desenvolvimento de 
    sistemas. Trata-se de uma aula argumentativa feita nos mesmos moldes da 
    anterior. O tema, contudo, era, aqui, outro: \textit{criogenia} e 
    \textit{eutanásia}, no contexto da discussão acerca da ciência 
    contemporânea. 
    
    Novamente, a sala demora cerca de dez minutos para estar organizada, 
    numa sombra do pátio externo, em forma de roda. Délcio distribuiu as 
    folhinhas contendo as questões e quem inicia o debate é Tamires. Ela indaga 
    a Clayton, um dos alunos mais tímidos da sala, o que está contido em seu 
    papel --- ``Se você tivesse a opção de se congelar para depois, no futuro,  
    tornar-se imortal, você o faria?''. 
    
    Ao que Clayton tenta, entre interrupções de seus colegas, que o caçoam por 
    ser excessivamente acanhado, responder que sim. Mas fica envergonhado em 
    justificar sua resposta. Após alguma insistência por parte de Délcio, 
    o rapaz diz algo que, a bem da verdade, foi ininteligível. Para não quebrar 
    o fluxo da aula, resolveu-se escolher outro aluno. Apontaram Lorena, que 
    indaga a Raphael ``Se você tivesse uma doença crônica, ainda assim 
    você optaria pela criogenia, sabendo que há pouca chance de cura, mesmo 
    no futuro?''. Raphael fica pensativo. Após alguns segundos dispara que 
    ``sim, pois preferia arriscar essa chance do que ser simplesmente 
    esquecido para todo o sempre.'' 
    
    Nisto, Carlos, entra na discussão dizendo que preferia a eutanásia para 
    estes casos. Délcio, estupefato, afirma que uma coisa não exclui a outra, 
    e que alguém pode sim optar por passar por eutanásia e depois ser congelado. 
    Muitos alunos ficam perplexos. Carlos procura se justificar citando que 
    não quer ter qualquer tipo de sobrevida após sua morte: quer que sua 
    família passe pelo que considera ser um natural processo de luto e que, 
    além disso, o processo criogênico é bastante caro, inacessível a pessoas 
    com poucos recursos. 
    
    Délcio elogia a resposta de Carlos e pontua a relevância do processo de 
    luto para o ser humano, dando aos familiares a chance, talvez, de 
    ``sentir a perda e não a falta.'' Pede, então, para Carlos escolha o próximo 
    aluno  a fazer uma pergunta. Em seguida, continuaram até o fim do 
    encontro no tema da eutanásia. Alguns alunos comentaram ter certo ``medo'' 
    da morte e que a criogenia lhes parecia uma saída viável para, talvez, 
    ``contarem o que viram''. Fora isso, as intervenções foram mais ou menos 
    todas no mesmo grau já observado. 
    
    \subsection*{Aula III}
    
    Terceiro encontro aqui relatado: dia 26/09, uma aula expositiva sobre 
    filosofia da ciência e Aristóteles, ministrada para o 1º AQI (primeiro ano
    do ensino médio integrado ao curso de técnico de química). Délcio faz a 
    chamada e em pouco mais de cinco minutos, inicia a aula, depois de dispor 
    no quadro negro, os pontos chave do conteúdo. 
    
    É uma aula clássica, conteudista. Délcio cita algumas passagens do livro 
    Física, Meterologia e do Organon e parte delas para explicar a busca na 
    filosofia pela lógica. Alguns alunos fazem expressão de dúvida, mas uma boa 
    parte da sala está atenta. Uns poucos discentes, ao fundo da sala, 
    optam francamente por não prestar atenção, mas não atrapalham. A aluna 
    Mariana foi a única a levantar a mão, ao fim da explicação e disse não 
    compreender bem a ideia de ``materia''. Délcio optou por explicar a ela 
    após o término da aula. 
    
    Findada a explicação, que levou cerca de dezessete minutos, Délcio 
    distribuiu uma folha contendo alguns excertos de textos clássicos de 
    Aristóteles e algumas perguntas de interpretação de texto. A ideia é que 
    os alunos começassem esta atividade e, caso não terminasse, a entregassem 
    no próximo encontro. 
    
    \subsection*{Aula IV}

	Numa terça-feira, Délcio, pretendia passar, ainda no 1º DS, uma atividade 
	sobre filosofia da ciência e Aristóteles, como tratado no encontro 
	anterior. A turma, então relatou ter tido certa dificuldade com o conteúdo, 
	e muitos ainda estavam entregando a atividade começada na aula anterior. 
	Os alunos pediram uma pequena revisão deste conteúdo e, também, dos 
	anteriores.  
	
	Apesar de ser um docente bastante respeitado na Etec, neste encontro foi 
	pego no contrapé, pois não conseguira, a contento explicar o tópico à 
	turma. Contrariado, fez da atividade um trabalho para casa e fez a chamada. 
	
	\newpage
		
	\section{O curso em vista dos parâmetros nacionais}
	
    
	
	\newpage
	
	\section{O curso ante a bibliografia desta disciplina}
	

	
	
	\newpage
	
	\section{Epílogo}
	
    
	
	\newpage
	

	
	\newpage
	
	\begin{thebibliography}{9}
		\bibitem{adorno}
		ADORNO, T. 
		\textit{Education after Auschwitz. }   
		\\\texttt{https://www.ime.usp.br/~tadeu/EDF0285/A10\_Adorno.pdf}
		
		
	\end{thebibliography}

\end{document}

