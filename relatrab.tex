\documentclass[12pt,a4paper]{article}
\usepackage[utf8]{inputenc}
\usepackage[portuguese]{babel}
\usepackage{amsmath}
\usepackage{amsfonts}
\usepackage{amssymb}
\usepackage{graphicx}
\usepackage{parskip}
\usepackage{setspace}
\usepackage{scrextend}
\usepackage{titling}
\usepackage{xcolor}
\usepackage{epigraph}
%\usepackage{draftwatermark}
%\SetWatermarkText{Em progresso}
%\SetWatermarkScale{3}

\newenvironment{citac}{\begin{addmargin}[4cm]{1em} \footnotesize}{\normalfont \end{addmargin}}

\usepackage[left=3.00cm, right=2.00cm, top=3.00cm, bottom=2.00cm]{geometry}
\author{Pedro T. R. Pinheiro}
\date{São Paulo\\2020}
\title{Relatório de Estágio}

\newcommand{\subtitulo}{A filosofia por ensino remoto}
\newcommand{\disciplina}{EDF0424 - Metodologia do Ensino de Filosofia II}
\newcommand{\departamento}{Departamento de Metodologia do Ensino}
\newcommand{\unidade}{FE - Faculdade de Educação}
\newcommand{\prof}{Paulo H. F. Silveira}

\begin{document}
	\begin{center}
				\textbf{
				\LARGE USP - UNIVERSIDADE DE SÃO PAULO \\
			}
			\Large \textsc{\unidade} \\
			\large \textsc{\departamento}\\
			\vspace*{1cm}
				
			Disciplina: \disciplina; \\Prof.: \prof
			\vfill
			\begin{center}
				{\Large \textsc{\theauthor}} \\ 
				\vspace{1cm}
				\LARGE\textbf{\thetitle} \\
				\Large\emph{\subtitulo}
			\end{center}
			\vfill
			\large\thedate
			\vspace*{1cm}
			\thispagestyle{empty}			
	\end{center}

	\newpage

	\widowpenalty10000
	\clubpenalty10000
	\setlength{\parskip}{0.5cm}
	\setlength{\parindent}{1.1cm}
	\onehalfspacing
	
	\section{Preâmbulo}

	\subsection{Sobre este ensaio e seu escopo}

	A discussão presente neste ensaio engloba as experiências de ensino 
	à distância proporcionadas pelo Centro de Mídias do Estado de 
	São Paulo (CMSP) e da Secretaria da Educação do Estado do Paraná 
	(SEED). No primeiro caso, fora acompanhado o docente Marconi F. L. 
	Cabral, M. F. Degan e L. Gonçalves, enquanto, no segundo, 
	acompanhara-se 	R. Polesi e S. dos Anjos. Como sempre, o foco é fazer 
	uma análise 	da estrutura do curso, como os materiais e recursos 
	disponíveis são utilizados e, também, a resposta das turmas envolvidas 
	às estratégias 	correntemente adotadas. 
	
	Há de se frisar o imenso abismo entre as discussões realizadas no curso 
	e a dita ``realidade'': se por um lado temos, dentro dos muros da 
	universidade, a possibilidade de discutir métodos, ali, na prática, 
	defronte a uma sala de aula, prevalece o que coloquialmente poder-se-ia 
	rotular de ``vai como dá''. Como é de se esperar, a parte final deste 
	relatório procura, dentro da medida do possível, fechar este abismo. 
	
	Fechar este abismo -- o que, em si, já constitui numa tarefa hercúlea --
	é, contudo, meramente um primeiro passo. Com isto é 
	apenas possível diagnosticar algumas mazelas. E é ainda maior desafio 
	propor soluções. Não se desvela, ao jovem professor, de pronto, 
	a vasta gama de ferramentas para se trabalhar com os mais diversos 
	tipos de classes. Pelo contrário, são, inicialmente, muito limitados 
	os recursos com que conta o licenciado neófito. Deste modo, seria 
	demasiado ambicioso esperar deste relatório uma solução efetiva 
	às imperfeições observadas. 

	Logo, é apenas isto: uma tentativa. Tentativa de compreensão e 
	tentativa de solução. Afinal, sob certo ponto de vista, por toda a 
	volatilidade encontrada, e por toda a sua incerteza relacionada a 
	ser uma ciência que lida diretamente com os nervos da natureza 
	humana, não será ela inevitavelmente e quase exclusivamente empírica? 
	
	\newpage
	
	\section{Materiais e Organização do Curso}

	Em razão da crise sanitária, fora adotado, em ambos os casos, o 
	modelo de aula remota, com todos os seus percalços e limitações, 
	que serão discutidas mais adiante. A plataforma escolhida fora o 
	YouTube, serviço fundado em 2005 e comprado pela Google poucos 
	anos depois, consolidado como a mais importante plataforma de 
	streaming gratuita da internet. A CMSP ministrou seu curso, 
	primariamente, de maneira síncrona (isto é, ao vivo), ao passo de 
	que a SEED o fez de maneira assíncrona. 

	Observa-se que o curso da SEED se atém a uma estrutra relativamente 
	tradicional: em diversas aulas, observa-se um formato de curso de 
	história da filosofia. O serviço Google Classroom é usado como forma 
	de prover mecanismos de presença, avaliação e disponibilização de 
	material. Infelizmente, não foi possível
		

    	
	\newpage
	
	\section{As aulas}

	\subsection{SEED - 1º Ano EM - 6 de agosto}

	\textbf{Tema:} Copérnico e o Heliocentrismo

	Ministrada no começo de agosto, é a trigésima-quarta aula do curso. 
	Neste encontro, Polesi faz uma exposição histórica do tema. Seu 
	ponto de partida é contexto histórico, i.e. quem fora Copérnico, 
	qual fora sua contribuição para o pensamento científico. Em seguida, 
	em um slide, é proposta a seguinte questão: “O que significa revolução 
	copernicana?” 

	O professor confere aos alunos ao menos cinco minutos para tentar 
	formular uma resposta, antes de apresentar sua própria resposta à 
	questão, no slide seguinte. Daí em diante, procede com a explicação 
	acerca do método copernicano. Faz também uma exposição das influências 
	intelectuais de copérnico, e, para garantir que os alunos lembrariam 
	do conteúdo, propõe uma questão para ajudá-los nas anotações: “Qual 
	foram as influências de Copérnico em seu pensamento?”, à qual precisam 
	responder em cinco minutos. 

	Em seguida, apresenta, novamente, a resposta correta, pontificando as 
	ideias principais da seção (influências realistas e neoplatônicas). 
	Sem mais delongas, prossegue, então, continua em direção ao próximo 
	assunto, explorando mais profundamente a situação da astronomia 
	antes de Copérnico e como Copérnico instaura uma quebra paradigmática 
	em relação à colcha de retalhos que era a teoria astronômica outrora. 

	Encerrando este passo, há uma outra questão: “Por que as respostas da 
	teoria ptolomaica deveriam ser superadas?”, convidando, logo, os 
	estudantes a recapitular os motivos pelos quais Copérnico rompera com 
	a “tradição” nos estudos em astronomia. Cinco minutos depois, Polesi 
	retorna, comentando sobre a ausência à dita teoria ptolomaica de um 
	“corpo fundante”, elemento crucial a uma teoria científica. 

	Neste momento, se aproxima ao final da aula, e o docente avança à 
	análise das consequências do pensamento copernicano e de seu legado 
	para a modernidade, tendo influenciado Newton, Descartes, Kepler 
	(que continuou seu trabalho) e outros. Para sumariar, é posta, então, 
	a última pergunta do encontro: “qual foi a principal contribuição de 
	Copérnico?“, o que parece convidar os alunos a revisitar com mais 
	profundidade todo o conteúdo exposto neste dia. 

	Polesi encerra o encontro lendo rapidamente a resposta para a questão 
	proposta e expõe alguns slides contento referências bibliográficas 
	interessantes aos interessados em se aprofundar no assunto. 

	\subsection{CMSP - 1º Ano EM - 13 de agosto}

	lorem ipsum dolor sit amet et volupat

	
	
	\newpage
		
	\section{Análise dos cursos}
	
	A “dinâmica” de uma aula remota é, em geral, radicalmente diferente 
	daquela observada em um encontro presencial, sem sombra de dúvidas. 
	As limitações e possibilidades das aulas à distância são assuntos da 
	ordem do dia, em tempos em que, é verdade, não é possível fazer de 
	qualquer outra maneira.

	Se educação à distância não é novidade alguma -- já sendo, inclusive, 
	estopim de infindáveis debates entre educadores desde que o avanço das 
	telecomunicações permitira que praticamente qualquer um tivesse acesso 
	a este tipo de ensino --, é preciso lembrar que sua introdução ao 
	ambiente escolar tradicional tem sido apontada como potencialmente 
	disruptiva já há algum tempo

	[Encontrar alguma citação para justificar isso]

	Logo, se, por um lado, se elimina o deletério efeito da algazarra 
	promovida pelos alunos ou o tempo gasto com organização da sala de 
	aula, por outro há uma seara de novos desafios a serem enfrentados 
	pelos docentes. Se no ambiente presencial é difícil controlar a 
	participação dos alunos, no ambiente virtual é, por outro lado, 
	garantir que ela exista de maneira efetiva. 

	O modelo adotado pelas instituições aqui acompanhadas não comportaria 
	parece, em ambos os casos, não comportar a participação dos alunos por 
	vídeo. No caso do curso oferecido pelo CMSP, a participação se dá via 
	texto: os alunos respondem, em um curto período (variando de um a dois 
	minutos) uma ou duas questões postas em um slide, e as respostas 
	balizam, posteriormente, o tom da aula. 
	


---


	Não há, a priori, nada de absolutamente peculiar na abordagem feita por 
	Délcio. Com exceção dos debates realizados ao ar livre, observa-se que 
	os encontros seguem uma estrutura bastante comum: preenche-se a lousa, 
	expõe-se alguns tópicos e tenta-se dar uma informação condensada e 
	minimamente útil num quadro de tempo reduzido, procurando não extrapolar 
	o \textit{attention span} típico do estudante secundarista. Este 
	\textit{attention span} parece menor a cada dia e fica a impressão de que 
	no calor da batalha, sacrifica-se a riqueza de conteúdo e alguma 
	possibilidade de discussão pós-conteúdo em nome da transmissão de um 
	mínimo de ideias. 
	
	Há um programa a ser seguido, e uma noção geral de que o conteúdo 
	de história da filosofia é parte de uma cultura geral que o aluno tem de 
	absorver, de alguma forma --- ``os conteúdos designam o conjunto de 
	saberes ou formas culturais cuja assimilação e apropriação pelos alunos e 
	alunas se considera essencial para seus desenvolvimentos e socialização.'' 
	(\cite{obiols}, p. 68). Dentro de uma perspectiva mais ``tradicional'', 
	aprender a filosofar, paralelo ao desenvolvimento como cidadão e 
	indivíduo, depende de alguma cultura prévia. Para que a discussão 
	filósofica funcione, é preciso que haja um conteúdo a ser discutido. 
	
	Há, portanto, boas razões para o conteudismo. Se a forma como transmite-se 
	o conteúdo é eficaz e suficiente à suscitação de discussões filosóficas, é 
	algo que ainda está por se ver. A observação dos esporádicos debates ao ar 
	livre sugere que não. Os alunos não parecem ter sido, até o momento, 
	capazes de conectar as ideias comentadas brevemente em sala com elementos 
	do ``mundo real''. Suas respostas e opiniões ainda são muito baseadas em 
	achismos e experiências pessoais. A priori, as aulas de filosofia parecem 
	influir muito pouco nos alunos. 
	
	Por outro lado, este tipo de influência só pode ser medido ao longo de 
	alguns bons anos, é uma influência a longo prazo. Um dia, muito tempo 
	depois, você lembra de um nome que ouviu falar na escola, pesquisa mais a 
	respeito 	e este talvez seja o principal trunfo de uma 
	``cultura geral'': dar as referências para que, no futuro, você mesmo seja 
	capaz de encontrar o caminho das pedras depois. 

	Para que a coisa funcione desta maneira, sem sobra de dúvidas importa que 
	haja uma avaliação cobrando ao menos uma parte destes conteúdos, o que a 
	avaliação usada por Délcio parece fazer de maneira satisfatória. Alguma 
	informação estes alunos hão de reter, de qualquer maneira. É isto, contudo, 
	suficiente? 
	
	É difícil dizer. O aluno do qual se trata aqui é um adolescente, no final 
	de sua vida escolar. Idealmente, todos deveriam ter instrumental 
	suficiente para ``saber usar'' esta cultura geral que adquirem na escola. 
	A realidade, contudo, é que nunca adquirem esta maturidade a tempo. De 
	modo que tudo o que a escola repassa vira um grande bloco inexpugnável de 
	informações desconexas e relações confusas. Quando a um aluno as ideias 
	se dispõe desta maneira, até a noção de uma cultura geral está ameaçada. 
	
	Ou seja, é de se imaginar que não só deve se ensinar o aluno filosofia, 
	história e matemática, mas também deve ser feito algum esforço no sentido 
	de ensiná-lo a \textit{filosofar}. Obiols admite que dentro desta 
	perspectiva é possível acabar entrincheirado numa escola filosófica, posto 
	que a forma de filosofar é instrinsicamente ligada à escola filósofica à 
	qual se adere (\cite{obiols}, p. 70). Mas se o custo de dar ao aluno 
	instrumentos para pensar, inferir, conjecturar é assumir alguma escola 
	de pensamento, talvez não seja um mal negócio. 
	
	Délcio parece evitar tal escolha. Suas aulas são, em grande parte das 
	vezes, genéricas como boa parte das aulas de filosofia. O pensamento 
	vira matéria morta. É como se outrora Descartes escrevesse em suas 
	Meditações uma série de coisas, mas que é isto... São apenas coisas 
	escritas --- que ficaram num distante passado. Ao expor o aluno ao 
	pensamento de um autor, talvez não seja má ideia dar a corda, deixar 
	que ele leia o livro, que se ``contamine'' pelas ideias do autor. 
	
	Ainda que não tenho o aluno o ímpeto de ler um livro --- é de se 
	esperar que, sobretudo em tempos de intenet e informações rápidas, 
	um adolescente não tenha a paciência de se dar ao árduo trabalho da 
	leitura ---, o professor ainda pode selecionar trechos, recortes 
	e deixar que o aluno veja razões para acreditar naquelas ideias. 
	Não há outra via de entrada para a filosofia. Não se a queremos viva. 
	
	É apenas um leve desvio em relação ao que o docente já faz, e parece 
	se encaixar no projeto já existe (uma vez que o docente já faz algumas 
	atividades com a turma que fogem à habitual aula expositiva, como já 
	fora mencionado aqui algumas tantas vezes) não seria de significativa 
	dificuldade. Isto, no entanto, requer abandonar um pouco a ``linearidade'' 
	do conteúdo e, em parte, o próprio conteúdo, se levarmos em conta o 
	reduzido tempo em sala de aula. 
	
	Não nos apeguemos, todavia, tanto ao currículo. Ele é importante, como 
	já se viu pouco acima, mas de pouco adianta tê-lo sem ter os 
	instrumentos para manipulá-lo. Cabe ao professor dosar quanta ``praxis'' 
	colocará em suas aulas e quanto conteúdo por trás delas estará. 	
	Para A. Cerletti, a ideia de um currículo determinado, rígido, já é, 
	de saída, uma ``ficção'' (\cite{cerletti}, p. 79) e que o docente tem 
	papel ativo em formatar o currículo da maneira mais adequada: 
	
	\begin{citac}
		(...) nunca é possível ``aplicar'' ou pôr em execução um currículo 
		sem a intervenção criadora do docente, que é quem deverá assumir as 
		condições ``reais'' do ensino. No caso da filosofia, isso é 
		particularmente significativo. Se ensinar filosofia implica ensinar 
		a filosofar, deve-se esperar que sempre de quem ``aprende'' a 
		intervenção ativa no perguntar filosófico e na busca de respostas, e 
		isso não se pode levar adiante a não ser sob certas condições que o 
		professor deverá poder viabilizar. \cite{cerletti}, p. 79
	\end{citac}
	
	A sugestão do PCN quanto a este ponto sugere, sutilmente, um curso de 
	ação: 
	
	\begin{citac}
	(...) de um ponto de vista propedêutico, a conexão interna entre conteúdo e método deve tornar-se evidente: que o estudante tenha se apropriado significativamente de um determinado conteúdo filosófico significa, ao mesmo tempo, que ele se apropriou conscientemente de um método de acesso a esse conteúdo.  
	
		Apropriar-se do método adequado significa, primariamente, portanto, construir e exercitar a \textbf{capacidade de problematização}. Nisto consiste, talvez, a contribuição mais específica da Filosofia para a formação do aluno do Ensino Médio: auxiliá-lo a tornar temático o que está implícito   e   problematizar   o   que   parece   óbvio.   Portanto,   a   competência   de   \textbf{leitura significativa}   de   textos   filosóficos   consiste,   antes   de   mais   nada,   na   capacidade   de   problematizar o que é lido, isto é, \textbf{apropriar-se reflexivamente} do conteúdo. 
		\cite{pcn}, p. 50
	\end{citac} 
	
	Quando fala-se nesta tal de \textit{apropriação reflexiva}, já se 
	sugere, de certa forma, que uma possível maneira de apresentar o curso 
	seja tornar algumas ideias passíveis de apropriação por parte dos 
	estudantes. E, como o trecho ainda professa --- ``problematizar o que é 
	óbvio'' ---, ``confrontar'' as ideias trabalhadas com a ``realidade''. 
	
	Tornando aos debates mediados por Délcio, pode-se dizer que são uma 
	tentativa justamente de levantar este tipo de discussão. Mas, para que 
	isso suceda, é preciso que a aula não pareça um organismo tão desconectado 
	desta atividade que, aos alunos, passa mais como um entretenimento, ou, 
	coloquialmente, uma forma de ``quebrar o gelo'' e dar-lhes a chance de 
	respirar um ar diferente do viciado ar da sala de aula. 
	
	Desta maneira, 
	
	\newpage
	
	\section{Sobre a proposta didática}
	
	\newpage
	
	\section{Epílogo}
	
	\newpage
	
    \begin{thebibliography}{9}
		% bibtexar isto! 
		
		\bibitem{pcn} 
		BRASIL. 
		\textit{Parâmetros Curriculares Nacionais}.  
		Ensino Médio: IV. Ciências Humanas e suas Tecnologias. 
		MEC/SEF, 1998. 
		
		\bibitem{obiols}
		OBIOLS, Guillermo. 
		\textit{Una Introducción a la Enseñanza de Filosofía}. 
		FCE, Buenos Aires, 2002. 
		
		\bibitem{cerletti}
		CERLETTI, Alejandro. 
		\textit{O ensino de filosofia como um problema filosófico}. 
		Autêntica Editora, Belo Horizonte, 2009. 

	\end{thebibliography}
	


\end{document}

