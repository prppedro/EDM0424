\documentclass[12pt,a4paper]{article}
\usepackage[utf8]{inputenc}
\usepackage[portuguese]{babel}
\usepackage{amsmath}
\usepackage{amsfonts}
\usepackage{amssymb}
\usepackage{graphicx}
\usepackage{parskip}
\usepackage{setspace}
\usepackage{scrextend}
\usepackage{titling}
\usepackage{xcolor}
\usepackage{epigraph}
%\usepackage{draftwatermark}
%\SetWatermarkText{Em progresso}
%\SetWatermarkScale{3}

\newenvironment{citac}{\begin{addmargin}[4cm]{1em} \footnotesize}{\normalfont \end{addmargin}}

\usepackage[left=3.00cm, right=2.00cm, top=3.00cm, bottom=2.00cm]{geometry}
\author{Pedro T. R. Pinheiro}
\date{São Paulo\\2019}
\title{Relatório de Estágio}

\newcommand{\subtitulo}{O ensino de filosofia na Etec Raposo Tavares}
\newcommand{\disciplina}{EDF0423 - Metodologia do Ensino de Filosofia}
\newcommand{\departamento}{Departamento de Metodologia do Ensino}
\newcommand{\unidade}{FE - Faculdade de Educação}
\newcommand{\prof}{Paulo H. F. Silveira}

\begin{document}
	\begin{center}
				\textbf{
				\LARGE USP - UNIVERSIDADE DE SÃO PAULO \\
			}
			\Large \textsc{\unidade} \\
			\large \textsc{\departamento}\\
			\vspace*{1cm}
				
			Disciplina: \disciplina; \\Prof.: \prof
			\vfill
			\begin{center}
				{\Large \textsc{\theauthor}} \\ 
				\vspace{1cm}
				\LARGE\textbf{\thetitle} \\
				\Large\emph{\subtitulo}
			\end{center}
			\vfill
			\large\thedate
			\vspace*{1cm}
			\thispagestyle{empty}			
	\end{center}

	\newpage

	\widowpenalty10000
	\clubpenalty10000
	\setlength{\parskip}{0.5cm}
	\setlength{\parindent}{1.1cm}
	\onehalfspacing
	
	\section{Preâmbulo}
	
	Este breve ensaio versa sobre a experiência de estágio na Etec Raposo Tavares. A mesma escola cujo semestre letivo fora acompanhado durante o primeiro módulo da disciplina de Metodologia do Ensino de Filosofia. Desta vez, entretanto, o trabalho restringe-se a apenas um docente: Délcio. Como o foco desta disciplina, no segundo semestre, é principalmente o de auxiliar o aluno na construção de um plano de curso coeso e sintonizado com a realidade da escola, opta-se aqui por acompanhar apenas um professor com formação na área, o que facilitou a organização para algumas das regências, experiência a qual é, a certo modo, de imprescindível importância à feitura de um primeiro plano de curso. 
	
	Uma considerável fatia das horas de estágio fora, entretanto, constituída mais de observação dos encontros e esparsas intervenções 
	
	\newpage
	
	\section{Materiais e Organização do Curso}
	

	
	\newpage
	
	\section{As aulas}
	

	
	\newpage
		
	\section{O curso em vista dos parâmetros nacionais}
	

	

	
	\newpage
	
	\section{O curso ante a bibliografia desta disciplina}
	

	
	
	\newpage
	
	\section{Epílogo}
	

	
	\newpage
	

	
	\newpage
	
	\begin{thebibliography}{9}
		\bibitem{adorno}
		ADORNO, T. 
		\textit{Education after Auschwitz. }   
		\\\texttt{https://www.ime.usp.br/~tadeu/EDF0285/A10\_Adorno.pdf}
		
		
	\end{thebibliography}

\end{document}