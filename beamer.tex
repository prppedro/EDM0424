\documentclass[12pt]{beamer}
\usepackage[utf8]{inputenc}
\usepackage[T1]{fontenc}
\usepackage{lmodern}
\usepackage{enumerate}
\usetheme{CambridgeUS}
\begin{document}
	\author{Pedro T. R. Pinheiro (8983332)}
	\title{Um Plano de Curso}
	%\subtitle{}
	%\logo{}
	%\institute{}
	%\date{}
	%\subject{}
	%\setbeamercovered{transparent}
	%\setbeamertemplate{navigation symbols}{}
	\begin{frame}[plain]
	\maketitle
\end{frame}

\begin{frame}
\frametitle{Preâmbulo}

	\begin{itemize}
		\item Das asperezas do ensino básico; 
		\item Impossibilidade de se ter uma fórmula universal; 
		\item Ceticismo para com a cátedra; 
		\item Formulação pragmática de um curso. 
	\end{itemize}

\end{frame}

\begin{frame}
\frametitle{Premissas básicas}

	\begin{itemize}
		\item Uma estrutura genérica; 
		\item Foco em aulas oficina, discussões; 
		\item Planejado para um bimestre/semestre (dezesseis aulas). 
	\end{itemize}

\end{frame}

\begin{frame}
\frametitle{As aulas: I a IV}

	\begin{itemize}
		\item Aula I: aula termômetro; 
		\item Aula II: construção de um diálogo; 
		\item Aula III-IV: exibição de um filme. 
		\item Sugestões para a quarta aula: Barry Lyndon, Ghost in the Shell, 
		Die Welle etc. 
	\end{itemize}

\end{frame}

\begin{frame}
\frametitle{As aulas: V a IX}

	\begin{itemize}
		\item Aula V: Interpretação do filme; 
		\item Aula VI-IX: Microsseminários. 
	\end{itemize}

\end{frame}

\begin{frame}
\frametitle{As aulas: IX, X}

\begin{itemize}
	\item Aula IX: exame sobre o conteúdo discutido nos seminários; 
	\item Aula X: devolutiva do exame. 
\end{itemize}

\end{frame}

\begin{frame}
\frametitle{As aulas: XII, XIII}

\begin{itemize}
	\item Aula XII-XIII: outro filme. 
\end{itemize}

\end{frame}

\begin{frame}
\frametitle{As aulas: XIV-XV (Avaliação)}

\begin{itemize}
	\item Avaliação; 
	\item Correção subsequente. 
\end{itemize}

\end{frame}

\begin{frame}
\frametitle{As aulas: XVI}

\begin{itemize}
	\item Aula XVI: Aula ``livre''
\end{itemize}

\end{frame}

\begin{frame}
\frametitle{As aulas: IX, X}

\begin{itemize}
	\item Aula IX: exame sobre o conteúdo discutido nos seminários; 
	\item Aula X: devolutiva do exame. 
\end{itemize}

\end{frame}

\end{document}