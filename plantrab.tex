\documentclass[12pt,a4paper]{article}
\usepackage[utf8]{inputenc}
\usepackage[portuguese]{babel}
\usepackage{amsmath}
\usepackage{amsfonts}
\usepackage{amssymb}
\usepackage{graphicx}
\usepackage{parskip}
\usepackage{setspace}
\usepackage{scrextend}
\usepackage{titling}
\usepackage{epigraph}
\usepackage{soul}
%\usepackage[activate={true,nocompatibility},final,tracking=true,kerning=true,spacing=true,factor=1100,stretch=10,shrink=10,spacing=nonfrench]{microtype}
\usepackage{xcolor}
\usepackage[left=3.00cm, right=2.00cm, top=3.00cm, bottom=2.00cm]{geometry}
\author{Pedro T. R. Pinheiro}
\date{2019}
\title{Um Plano de Curso}

%\usepackage{draftwatermark}
%\SetWatermarkText{Em progresso}
%\SetWatermarkScale{3}

\newenvironment{citac}{
	\begin{addmargin}[4cm]{1em} \footnotesize}{\normalfont \end{addmargin}
}

\newcommand{\subtitulo}{}
\newcommand{\disciplina}{EDF0424 - Metodologia do Ensino de Filosofia}
\newcommand{\departamento}{Departamento de Metodologia do Ensino}
\newcommand{\unidade}{FE - Faculdade de Educação}
\newcommand{\prof}{Paulo H. F. Silveira}

\begin{document}
	\begin{center}
				\textbf{
				\LARGE USP - UNIVERSIDADE DE SÃO PAULO \\
			}
			\Large \textsc{\unidade} \\
			\large \textsc{\departamento}\\
			\vspace*{1cm}
				
			Disciplina: \disciplina; \\Prof.: \prof
			\vfill
			\begin{center}
				{\Large \textsc{Pedro T. R. Pinheiro}} \\ 
				\vspace{1cm}
				\LARGE\textbf{\thetitle} \\
				\Large\emph{\subtitulo}
			\end{center}https://libgen.me/search/all?q=Matthew%20Lipman&adblock=true&year=0&search=Matthew%20Lipman&collection=all&currentPage=1&perPage=0
			\vfill
			\large São Paulo \\
			\large\thedate
			\vspace*{1cm}
			\thispagestyle{empty}
	\end{center}

	\newpage
	
	%\widowpenalty10000
	%\clubpenalty10000
	\setlength{\parskip}{0.5cm}
	\setlength{\parindent}{1.1cm}
	\onehalfspacing
	
	\section{Preâmbulo}
	
	%\setlength{\epigraphrule}{0pt}
	%\epigraph{\emph{``Como uma metrópole, o meu coração não pode parar / 
	% Mas também não pode sangrar eternamente''}}
	%{--- Belchior, \textit{Monólogo das Grandezas do Brasil}}
	
	A confecção de um curso para o ensino médio é menos trivial do que 
	pode, a priori, parecer. Da elaboração do material à produção do 
	espaço físico, é notável a necessidade de se levar em conta uma 
	série de elementos ambientais e sociais. E nem mesmo o mais completo
	estudo antropológico seria capaz de abarcar as mais plurais 
	configurações possíveis para uma dada turma. Logo, a feitura deste 
	dito plano de aula não é senão uma genérica descrição do que 
	parece a este autor de modo minimamente universal. É, por natureza, 
	um trabalho pouco prescritivo. 
	
	Dentre uma ampla gama de possíveis referências, opto por tentar, 
	na medida possível, encontrar o caminho que pareça ser mais 
	genérico e universal possível. Assim, servirá, de certo modo, mais como 
	um \textit{framework} do que como um curso em si. Pretende-se 
	elaborar um esqueleto onde podem se encaixar técnicas e modelos de 
	aula diversos. Claramente, haverá, uma certa necessidade de optar, 
	ao fim e ao cabo, por uma solução completa, mas há de ser preocupação 
	posterior. 
	
	Comumente, há, no jovem estudante do ensino básico, uma tendência a 
	voltar-se a assuntos de escopo bastante reduzido, frequentemente sob 
	a égide de seu próprio indivíduo. Nem todo o ser humano é, entretanto, igual, 
	como, por mesma lógica, nenhuma sala de ensino médio se porta de mesma 
	maneira que outra. Cada caso é um caso. Ora, não seria sensato pensar 
	o curso com base na receptividade dos indivíduos? 
	
	\textbf{TIRAR DO RABO UMA CITAÇÃO QUE CORROBORE ISSO}
	
	TERMINAR ESTA SECAO DEPOIS
	
	\newpage
	
	\section{Plano de Aulas}
	
	Em primeiro lugar, para se ministrar um curso, é preciso pensar na 
	estrutura de aulas, levando em conta tempo e espaço. O primeiro 
	princípio a ser adotado aqui é que será usada a primeira aula 
	como uma espécie de ``termômetro'' da sala. Isto é, não se ministra 
	conteúdo algum, tampouco se realiza qualquer atividade que seja parte 
	da ``programação'' normal. Isto se deve ao fato de que para ministrar 
	uma aula é preciso conhecer os vícios e virtudes da turma, a fim de 
	atingir o maior número de alunos com o conteúdo programático. 
	
	Sugere-se que a primeira coisa se fazer neste encontro seja 
	questioná-los acerca do que é filosofia e qual sua importância. 	
	Uma possível forma de fazê-lo é evocar trechos e conceitos 
	clássicos da filosofia, como por exemplo aqueles encontrados em 
	Platão (\textit{Mito da Caverna}) e Kant (\textit{O que é esclarecimento?}). 
	Tal sugestão é feita pelos próprios Parâmetros Curriculares 
	Nacionais (PCN): 
	
	\textbf{CITAR A PORRA DO PCN, p.45-46}
	
	Deste debate é possível ter ao menos alguma ideia de qual a 
	``responsividade'' da turma para com certos conteúdos e ideias, 
	bem como o amadurecimento da turma para participar das aulas 
	e discussões. Isto é que dará respaldo para a organização de 
	todo o conteúdo subsequente do curso. Será possível saber, dentre 
	outras coisas, qual o \emph{attention span} da turma, suas 
	pré-disposições a interagir com os temas propostos e a forma 
	como absorvem estes temas.  
	
	É possível também fazer um exame das possíveis dificuldades 
	que serão encontradas neste percurso. 
	
	
	\newpage
	
	\begin{thebibliography}{9}
		% bibtexar isto! 
		
		\bibitem{brd} 
		BOURDIEU, Pierre. 
		\textit{A Escola Conservadora: as desigualdades frente à escola e a cultura}. 
		In: \textit{Escritos de Educação} 
		Editora Vozes, Petrópolis, 2003.
		
	\end{thebibliography}
	
	
	
\end{document}
