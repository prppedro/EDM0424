\documentclass[12pt,a4paper]{article}
\usepackage[utf8]{inputenc}
\usepackage[portuguese]{babel}
\usepackage{amsmath}
\usepackage{amsfonts}
\usepackage{amssymb}
\usepackage{graphicx}
\usepackage{parskip}
\usepackage{setspace}
\usepackage{scrextend}
\usepackage{titling}
\usepackage{epigraph}
\usepackage{soul}
\usepackage{enumitem}
%\usepackage[activate={true,nocompatibility},final,tracking=true,kerning=true,spacing=true,factor=1100,stretch=10,shrink=10,spacing=nonfrench]{microtype}
\usepackage{xcolor}
\usepackage[left=3.00cm, right=2.00cm, top=3.00cm, bottom=2.00cm]{geometry}
\author{Pedro T. R. Pinheiro}
\date{2019}
\title{Um Plano de Curso}

%\usepackage{draftwatermark}
%\SetWatermarkText{Em progresso}
%\SetWatermarkScale{3}

\newenvironment{citac}{
	\begin{addmargin}[4cm]{1em} \footnotesize}{\normalfont \end{addmargin}
}

\newcommand{\subtitulo}{}
\newcommand{\disciplina}{EDF0424 - Metodologia do Ensino de Filosofia}
\newcommand{\departamento}{Departamento de Metodologia do Ensino}
\newcommand{\unidade}{FE - Faculdade de Educação}
\newcommand{\prof}{Paulo H. F. Silveira}

\begin{document}
	\begin{center}
				\textbf{
				\LARGE USP - UNIVERSIDADE DE SÃO PAULO \\
			}
			\Large \textsc{\unidade} \\
			\large \textsc{\departamento}\\
			\vspace*{1cm}
				
			Disciplina: \disciplina; \\Prof.: \prof
			\vfill
			\begin{center}
				{\Large \textsc{Pedro T. R. Pinheiro}} \\ 
				\vspace{1cm}
				\LARGE\textbf{\thetitle} \\
				\Large\emph{\subtitulo}
			\end{center}
			\vfill
			\large São Paulo \\
			\large\thedate
			\vspace*{1cm}
			\thispagestyle{empty}
	\end{center}

	\newpage
	
	%\widowpenalty10000
	%\clubpenalty10000
	\setlength{\parskip}{0.5cm}
	\setlength{\parindent}{1.1cm}
	\onehalfspacing	
	
	\section*{Preâmbulo}
	
	%\setlength{\epigraphrule}{0pt}
	%\epigraph{\emph{``Como uma metrópole, o meu coração não pode parar / 
	% Mas também não pode sangrar eternamente''}}
	%{--- Belchior, \textit{Monólogo das Grandezas do Brasil}}
	
	Muito discute-se acerca da educação, seus problemas, seus desafios, suas particularidades. É, de fato, a depender do contexto, um termo perigosamente vazio. Claramente, foge aos intentos deste autor forjar uma definição perene a este termo tão repleto de significados. É importante, no entanto, antes de se discutir \textit{didática}, pensar um pouco a ideia de educação \textit{per se}. Afinal, se a \textit{didática} é ação responsável por transmitir educação, o que torna alguém \textit{educado} é sem dúvidas um objeto a perpassar a didática. 
	
	Popularmente, fala-se muito em aprender uma lição. Geralmente quanto um sujeito aprende com erro próprio ou alheio. Mesmo considerando o sentido majoritariamente negativo da expressão, pode-se dizer que um dos motivos pelos quais a “lição” é tão bem aprendida é que, de certo modo, o sujeito recebe o resultado de ações compreendidas dentro de seu próprio projeto. 
	
	Projeto é propriamente o termo empregado por Meirieu. Esta ideia nos leva a pressupor certa necessidade de haver compatibilidade entre o conteúdo a ser transmitido e a “realidade” do aluno. É, de certa forma, um processo de tradução. O interesse do estudante pode ser obtido a partir do momento em que certa ideia cause certo impacto em seu mundo. Então assim como alguém aprende uma lição, lida com uma ideia muitas vezes adversa ou inesperada e com isso aprende. 
	
	Em grande parte das vezes, o processo de aprendizado mostra-se como uma série de choques, pequenos conflitos. Noutras, frequentemente por mera coincidência, uma ideia ou conteúdo ocorre de ser naturalmente compatível com o interesse do aluno. E, claramente, é preciso ressaltar: este processo não ocorre de forma igual a todos. Em alguns casos, a premissa deste parágrafo pode até mesmo inverter-se. 
	
	Trabalhemos, no entanto, com a média. Uma estratégia perigosa, reconhece-se, porém muitas vezes inescapável quando consideramos uma educação coletiva e democratizada. Se há alunos que respondem bem ao fluxo contínuo de informações ou, ainda, aqueles que se descobrem autodidatas, pouco importa, por ora. A maioria dos estudantes, entende-se, perece de problemas como falta de atenção e confrontação. Dubet, em comentário a uma experiência como professor de segundo grau, explica: “Aprendi que para uma aula que dura uma hora, só se aproveitam uns vinte minutos, o resto do tempo serve para ‘botar ordem’, para dar orientações.” (DUBET 1997, p. 223)
	
	Ainda, na mesma entrevista, Dubet afirma: “Os alunos são adolescentes completamente tomados pelos seus problemas de adolescentes e a comunidade dos alunos é ‘por natureza’ hostil ao mundo dos adultos, hostil aos professores.” (Idem, p. 225) Por este prisma, divisa-se um problema gigantesco: grande parte dos estudantes de ensino básico têm preocupações muito distintas e apartadas daquele mundo do qual é obrigado a fazer parte. A experiência há de ser, logo de saída, maçante. Há até mesmo uma forte possibilidade de bloqueio por parte do estudante. 
	
	Reformula-se, então, a questão: como ajudar este indivíduo a tornar-se educado? A resposta está na discussão. Mas uma discussão pode se dar de diferentes formas e repousar sobre diversos subterfúgios. A discussão aqui precisa empregar certa dose de pragmatismo: correr o risco de perder algumas informações na tradução do conhecimento para o projeto do aluno. Mais importaria a faísca de um conhecimento a ser explorado pelo aluno no futuro do que um punhado de informações decoradas, progressivamente apagadas com o passar do tempo. 
	
	Enfim, a confecção de um curso para o ensino médio é menos trivial do que 
	pode, a priori, parecer. Da elaboração do material à produção do 
	espaço físico, é notável a necessidade de se levar em conta uma 
	série de elementos ambientais e sociais. E nem mesmo o mais completo
	estudo antropológico seria capaz de abarcar as mais plurais 
	configurações possíveis para uma dada turma. Logo, a feitura deste 
	dito plano de aula não é senão uma genérica descrição do que 
	parece a este autor de modo minimamente universal. É, por natureza, 
	um trabalho pouco prescritivo. 
	
	Comumente, há, no jovem estudante do ensino básico, uma tendência a 
	voltar-se a assuntos de escopo bastante reduzido, frequentemente sob 
	a égide de seu próprio indivíduo. Nem todo o ser humano é, entretanto, 
	igual, como, por mesma lógica, nenhuma sala de ensino médio se porta de 
	mesma 	maneira que outra. Cada caso é um caso. Ora, não seria sensato 
	pensar o curso com base na receptividade dos indivíduos? 
	
	A ideia inicial deste ensaio era propor algo que funcionasse mais como 
	um \textit{framework} do que como um curso em si, isto é, 
	elaborar um esqueleto onde podem se encaixar técnicas e modelos de 
	aula diversos. Entretanto, para atender aos requisitos estabelecidos para 
	a produção deste trabalho, fora necessário ir um pouco além e desenvolver 
	a proposta de forma um pouco mais detalhada. 
	
	Dentro desta perspectiva, monta-se, aqui, um curso em dezesseis 
	encontros. O foco está na discussão, pois, a priori, sem conhecer a turma, 
	arrisco ser mais seguro apostar na participação do que contar com, no 
	mínimo, a passividade do aluno --- requisito mínimo para uma aula 
	expositiva no estilo tradicional. Especialmente considerando-se o contexto 
	da escola pública, em que estes desafios assumem dimensões ainda maiores. 
	
	
	\newpage
	
	\section{Plano de Aulas}
	
	Em primeiro lugar, para se ministrar um curso, é preciso pensar na 
	estrutura de aulas, levando em conta tempo e espaço. Assumiremos que 
	as turmas serão constituidas por cerca de 40 alunos, com aulas de 
	duração padrão 50 minutos e baixos recursos. Dada a quantidade de 
	pessoas na sala, deveremos elaborar aulas que capturem a atenção dos 
	mesmos e construír atividades de curta (geralmente uma aula) duração.
	Também gostaríamos de relacionar as metas do parâmetro nacional com 
	conteúdos culturais como filmes e livros não didáticos.	

	\subsection*{Primeria Aula}

	O primeiro princípio a ser adotado aqui é que será usada a primeira aula 
	como uma espécie de ``termômetro'' da sala. Aplicaremos uma sorte de 
	ficha diagnóstica, na forma de uma atividade de sala com o intuito 
	de medir quanto a turma está familiarizada com os temas e linguagem 
	da filosofia. Esta ficha não será uma avaliação ou um teste dos 
	conhecimentos da turma sobre o conteúdo do programa, mas uma roda de 
	conversa para medir a disponibilidade dos alunos de participar de 
	discussões e pensar a filosofia. 
	
	O modelo `roda de conversa' foi eleito 
	para nos distanciarmos do estigma que acomete a disciplina de filosofia:
	que se trata de uma área árida e de pouca relevância prática.
	Uma possível forma de fazê-lo é evocar trechos e conceitos 
	clássicos da filosofia, como por exemplo aqueles encontrados em 
	Platão (\textit{Mito da Caverna}) e Kant (\textit{O que é 
	esclarecimento?}) e em seguida fomentar a discussão e interpretação das 
	passagens citadas.

	Deste debate espera-se ser possível ter ao menos alguma ideia da 
	``responsividade'' da turma para com certos conteúdos e ideias, 
	bem como o amadurecimento da mesma para participar das aulas 
	e discussões. Isto é que dará respaldo para a organização de 
	todo o conteúdo subsequente do curso. Será possível saber, dentre 
	outras coisas, qual o \emph{attention span} da turma, suas 
	pré-disposições a interagir com os temas propostos e a forma 
	como absorvem estes mesmos -- bem como seus preconceitos sobre 
	a disciplina que teremos que tratar no decorrer do curso. 
	
	É possível também fazer um exame das possíveis dificuldades 
	que serão encontradas neste percurso. Alguns exemplos de possíveis 
	percalços seguem:

	\begin{enumerate}[label=\alph*)]
		\item	\label{prob:engaj} 
			Os estudantes não se engajarem com a discussão 
			dos textos.

		\item	\label{prob:ridic}
			Os estudantes, em geral, não se abrirem para a 
			revelar o que pensam por medo de serem ridicularizados
			ou repreendidos.

		\item	\label{prob:disc}
			Discussões infrutíferas e falta de foco.
	\end{enumerate}
	
	Para estas, a sugerimos para \ref{prob:engaj} que a discussão
	seja direcionada para incluír os possíveis interesses de alunos
	-- mixturando coisas correntes com questões filosóficas: por
	exemplo, relacionando certos video-jogos e outras narrativas 
	mais acercadas aos alunos, com as questões subjacentes que os 
	mesmos carregam consigo. 
	
	No caso que ocorra \ref{prob:ridic}, pouco pode ser feito senão
	acolher os alunos na sua subjetividade e garantir que qualquer 
	um que se expresse não seja açoitado pelo ridículo de seus 
	colegas. Além disso, exemplos de interpretações e motivações 
	podem estimular a participação.

	Quanto a \ref{prob:disc}, a orientação do professor é sempre
	necessária. Discutir é, em geral, frutífero pois o propósito
	do ensino de filosofia para o ensino médio encompassa o 
	desenvolvimento do espírito crítico e do pensamento claro nos 
	alunos -- coisa esta que se constroi em ato e na fala, com os 
	outros, ou consigo.

	\subsection*{Segunda Aula}

	Em havendo estabelecido o timbre da sala, podemos seguir no 
	projeto de desmontar certas ascepções sobre a filosofia 
	enquanto produzimos vontade de conhecimento nos alunos. Nesta
	etapa, e segunda aula, a intenção é montar uma roda de conversa
	em um sentido mais amplo. No que em uma aula anterior foi uma 
	exposição mais descontraída de questões da filosofia, faremos 
	uma mudança ainda mais radical: desorganizar a ordem da sala e 
	mudar o foco da discussão para as perguntas dos alunos sobre 
	a própria vida. 

	Este movimento tem triplo propósito: I) Aproximar o 
	professor ao aluno fisicamente pela disposição dos 
	estudantes -- que idealmente se daria um ambiente aberto como 
	uma ágora; II) Subverter a ideia que o professor 
	derramará conhecimento que deve ser absorvido, como se houvesse
	um eixo central na sala e o conteúdo imanasse do mesmo. 
	Substituíndo-o pelo círculo, onde nenhum ponto é favorecido e 
	a compreensão advém de qualquer e toda parte. III) 
	Construir nos alunos um respeito para além da autoridade efetiva
	do professor em sua função designada, como será visto a seguir.

	A atividade em si é uma conversa livre -- ainda que orientada 
	pelo professor. O docente se disporia a responder perguntas dos 
	alunos acerca da vida em seu sentido mais amplo, prosaico e 
	ingênuo. E -- também -- fazer perguntas que não se espera que 
	os alunos saibam ou consigam responder, para que se estabeleça 
	de maneira evidente o objetivo do curso: que os alunos passem 
	também a serem capazes de, sem o recurso de livros e referências,
	formular respostas com qualidade de raciocínio significativa 
	para questões que eles se disponham a formular sobre suas 
	respectivas realidades.
	
	Estratégias para a mediação da conversa: Para garantir que todos 
	se envolvam na conversa de alguma maneira, o professor pode, uma 
	vez reunidos os alunos, iniciar com uma questão e eleger um aluno 
	para tentar respondê-lo. O aluno escolhido, por sua vez pode 
	perguntar e escolher um outro para o responder ou, ainda, ceder a 
	palavra a alguém que queira fazê-lo. Neste molde, o professor 
	ainda pode escolher os mais acanhados entre os pupilos mas sem 
	ser um centro da conversa pois sempre cabe ao locutor a 
	prerrogativa de escolher seu sucessor.
	
	Dos problemas que podem ser encontrados temos dois principais:
	a indisponibildade do espaço de discussão, que precisa ser calmo 
	para a formação de ideias, também, calmas e estruturadas e -- 
	também -- a indisposição dos alunos. No caso indisposição dos 
	alunos uma medida de insistência e indignação teatral devem 
	solucionar a pouca vontade de falar dos alunos haja visto que é 
	impossível que eles não tenham opinão sobre nenhum assunto ou 
	sejam incapazes de forma-la e, certamente, podem raciocinar sobre 
	o mundo pois são munidos de intelecto.

	Os próprios impecílios encontrados podem servir de motivadores 
	para a conversa: se um aluno não permite que ela aconteça por 
	fazer barulho, há uma discussão que pode decorrer a respeito 
	de normas sociais, domesticação do corpo e mente, entre outros 
	tantos temas que uma mente aguçada aprende a encontrar nos 
	assuntos prosaicos da vida.

	\subsection*{Terceira e Quarta Aulas}
	
	Escolher-se-á um filme para exibir que tenha alguma relevância 
	para os temas das metas do programa. Durante o início da 
	exibição os alunos serão informados que deverão produzir um 
	trabalho de discussão de alguma questão que está presente no 
	filme, em formato livre, que de alguma forma elabore ou 
	explane ou contraponha questões ou problemáticas que julguem
	de interesse. 

	O objetivo desta atividade é duplo: por um, forçar os alunos 
	a praticarem o germe de pensamento autônomo que gostaríamos 
	de termos plantado nas aulas anteriores e, por outro, apresentar
	elementos de cultura geral de acesso mais restrito por 
	preconceitos (Barry Lyndon na lista de sugestões é um filme 
	que dura mais do que três horas) e desinteresse dos alunos para 
	que -- mesmo que os desinteresse -- conheçam.

	Entre o acervo que poderá ser exibido temos alguns filmes que 
	tem mérito de promover questões de ordem óbviamente filosófica:

	\paragraph{Ghost in the Shell} é uma animação 
		Japonesa década de 90 que aborda questões de 
		pessoalidade, existêncialismo, essência, 
		dualismo, transcendência, temas de política e 
		tecnologia no futurismo e distopia.

	\paragraph{O Show de Truman} que pode tratar tanto de 
		temas como vigilância, a natureza da realidade,
		absurdismo, alegoria da caverna.

	\paragraph{Barry Lyndon} que tem tanto a dizer sobre a 
		sociedade aristocrática, sobre a importância do
		acaso, do descontrole dos atores, falsa 
		civilidade, entre outros temas.

	\paragraph{Matrix} classicamente associado à alegoria 
		da caverna, poderia ser também interessante 
		fazer uma análise mais profunda das influências 
		das filosofias orientais que fundamentaram o 
		trabalho das Wachowskis. Elementos de Taoismo 
		poderiam adicionar um brilho a este tema já 
		muito conhecido.

	\paragraph{O Nome da Rosa} que rememora o tempo que 
		o conhecimento era intermediado e submetido ao 
		cerrado crivo da igreja, trata -- como queria 
		o seu autor -- de uma analogia a tempos modernos 
		mas igualmente tenebrosos que podem nos acometer.

	\paragraph{Akira} outra animação Japonesa de uma 
		sociedade em declínio e o perigo de se obter 
		poder incomensurável sem a capacidade e 
		maturidade de emprega-lo. Trata de juventude 
		marginalizada, contra cultura, jovens 
		delinquentes.

	\paragraph{Die Welle} ou ``A Onda''. O autoritativo 
		filme sobre a capacidade do ser humano de 
		produzir autoritarismo e fascismo, da banalidade
		da sua origem magnitude de seu desfecho.

	\paragraph{Chelovek s kino-apparatom} de Dziga Vertov e
		Elizaveta Svilova, filme metalinguístico com 
		elementos documentariais e experimentais. Obra 
		prima da década de 20.

	\paragraph{O Triúnfo da Vontade} Epítome da propaganda 
		Nazista, uma das obras máximas de Leni 
		Riefenstahl, excelente para produzir crítica e 
		repulsa nos alunos por ser um filme que busca 
		enaltecer -- pelos mesmos métodos que costumamos 
		empregar -- algo que fomos ensinados a abominar.
		
	Uma lista exaustiva é desnecessária e, sobretudo, 
	impossível de se elaborar em tempo hábil, mas não 
	ganharíamos muito com uma tal lista pois precisamos
	antes ler a sala para depois aplicar o filme.

	\subsection*{Quinta Aula}
	
	Interpretação do Filme --- Trabalho interpretativo do filme:
	Escolher um tema e/ou pergunta suscitada pela história ou 
	mídia e, a partir dela, desenvolver, na forma de uma redação, 
	em que o aluno desenvolve sua percepção sobre o tema. Trata-se 
	de uma atividade relativamente aberta. Que, entretanto, será 
	avaliada com um pouco mais de rigor, com vistas a garantir 
	que o discente ``mantenha o rumo'' da discussão. Ele terá de 
	tomar duplo cuidado para que sua redação não fuja nem à 
	discussão do filme e nem à do tema que esteja sendo tratado 
	no corrente momento. 

	\subsection*{Sexta e Sétima Aulas}
	
	Para o sexto encontro, organizar-se-á uma espécie de seminário. 
	O objetivo é garantir algum nível de ``engajamento'' por parte 
	do estudante. Este trabalho contempla diveras etapas, distribuídas 
	através de vários encontros. Para este primeira, espera-se que 
	os alunos se dividam em cerca de oito grupos (não muito mais do que 
	isso, por questões principalmente de tempo) e organizem-se para, 
	em cerca de dez minutos, fazer uma apresentação geral de algum
	filósofo. 
	
	Sugere-se que sejam apontados filósofos relativamente conhecidos 
	ao público de ensino médio. Alguns exemplos: Descartes, Hume, 
	Kant, Santo Agostinho, Platão, Nietzsche, Hegel e Locke. 
	
	\subsection*{Oitava e Nona Aulas}
	
	Os mesmos grupos preparam-se para, nesta aula, apresentar um 
	segundo seminário. Desta vez, entretanto, terão de ser mais 
	específicos. Escolherão uma obra do filósofo apresentado no 
	seminário anterior e farão, --- novamente --- em até dez minutos, 
	uma síntese desta obra e seus principais argumentos. 
	
	A título de exemplo, imagine-se que um grupo escolha apresentar, 
	digamos, um seminário acerca da \textit{Crítica da Razão Pura}, 
	de Kant. O grupo teria de explicar o contexto da obra (dogmáticos, 
	crise da razão, etc.), bem como algumas das soluções propostas 
	pelo pensador (doutrina transcendental dos conceitos etc.) Esta 
	apresentação não precisa ser avaliada com máximo rigor, mas quanto 
	mais próxima a descrição do grupo estiver da essência da obra, 
	tão melhor será o conceito atribuído ao final. 
	
	\subsection*{Décima e Décima Primeira Aulas}
	
	Aqui haverá a aplicação de um exame simples, contendo questões 
	de múltipla escolha, envolvendo conteúdos discutidos durante os 
	seminários. Alternativamente, pode-se pedir do aluno uma resenha 
	de dois ou três dos seminários. 
	
	Retorna-se, no encontro seguinte, aos alunos, todo o material 
	corrigido (ementas de seminário, resenhas, provinhas). Se possível, 
	procurar-se-á corrigir em sala o exame, que, quando for finalmente 
	devolvido ao estudante, deve estar idealmente comentado 
	pelo professor. 
	
	\subsection*{Décima Segunda e Décima Terceira Aulas}
	
	Torna-se aqui a assistir um filme. Pode ser um contido na 
	lista anterior ou não, mas tem de estar, na medida do possível, 
	alinhado com a temática principal que quer que esteja sendo 
	tratada no corrente momento. 
	
	Pode ser pedida alguma atividade acerca do filme. Mas, 
	levando-se em conta que, a esta altura, é bem provável que se esteja 
	já no fim do semestre e que o próximo encontro há de ser uma prova, 
	isto não parece conveniente nem para os alunos e nem para o professor, 
	que terá de gastar mais tempo corrindo estas atividades. 
	
	\subsection*{Décima Quarta e Décima Quinta Aulas}
	
	Avaliação em dois turnos. Nesta primeira aula do bloco, será aplicada 
	a prova final, contendo questões um pouco mais complicadas e 
	envolvendo todo o conteúdo até aqui tratado. A atividade é, 
	preferencialmente individual. 
	
	Na aula subsequente, os exames corrigidos (e, mais importantemente,  
	sem comentário algum do docente) são devolvidos aos seus alunos. Os 
	alunos deverão refazer as questões que erraram para que sejam 
    feitos os comentários finais. 
	
	\subsection*{Décima Sexta Aula}
	
	Em se tratando de ser uma aula de fim de semestre, após a avaliação, 
	pode ser uma oportunidade para melhorar o tratamento a alguns temas 
	discutidos, quer seja na forma de uma discussão ou de qualquer outra 
	atividade. 

	%\section{Termina, Porra Rodrigues Tadeu Pinheiro}
	%{\Huge\LaTeX}


	\newpage
	
	\begin{thebibliography}{9}
		% bibtexar isto! 
		
		\bibitem{brd} 
		BOURDIEU, Pierre. 
		\textit{A Escola Conservadora: as desigualdades 
		frente à escola e a cultura}. 
		In: \textit{Escritos de Educação} 
		Editora Vozes, Petrópolis, 2003.
				
		%\bibitem{jose} 
		%GOUDET. José. 
		%\textit{Opiniões Certas}. 
		%In: \textit{correspondência pessoal} 
		%Editora Vozes (Da Minha Cabeça), Petrópolis, 1998.

	\end{thebibliography}
	
	
	
\end{document}
