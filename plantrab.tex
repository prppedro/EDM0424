\documentclass[12pt,a4paper]{article}
\usepackage[utf8]{inputenc}
\usepackage[portuguese]{babel}
\usepackage{amsmath}
\usepackage{amsfonts}
\usepackage{amssymb}
\usepackage{graphicx}
\usepackage{parskip}
\usepackage{setspace}
\usepackage{scrextend}
\usepackage{titling}
\usepackage{epigraph}
\usepackage{soul}
\usepackage{enumitem}
%\usepackage[activate={true,nocompatibility},final,tracking=true,kerning=true,spacing=true,factor=1100,stretch=10,shrink=10,spacing=nonfrench]{microtype}
\usepackage{xcolor}
\usepackage[left=3.00cm, right=2.00cm, top=3.00cm, bottom=2.00cm]{geometry}
\author{Pedro T. R. Pinheiro}
\date{2020}
\title{Um Plano de Curso}

%\usepackage{draftwatermark}
%\SetWatermarkText{Em progresso}
%\SetWatermarkScale{3}

\newenvironment{citac}{
	\begin{addmargin}[4cm]{1em} \footnotesize}{\normalfont \end{addmargin}
}
\newenvironment{epigraf}{
	\begin{addmargin}[6cm]{1em} \scriptsize}{\normalfont \end{addmargin}
}

\newcommand{\subtitulo}{}
\newcommand{\disciplina}{EDF0424 - Metodologia do Ensino de Filosofia}
\newcommand{\departamento}{Departamento de Metodologia do Ensino}
\newcommand{\unidade}{FE - Faculdade de Educação}
\newcommand{\prof}{Paulo H. F. Silveira}

\begin{document}
	\begin{center}
				\textbf{
				\LARGE USP - UNIVERSIDADE DE SÃO PAULO \\
			}
			\Large \textsc{\unidade} \\
			\large \textsc{\departamento}\\
			\vspace*{1cm}
				
			Disciplina: \disciplina; \\Prof.: \prof
			\vfill
			\begin{center}
				{\Large \textsc{Pedro T. R. Pinheiro}} \\ 
				\vspace{1cm}
				\LARGE\textbf{\thetitle} \\
				\Large\emph{\subtitulo}
			\end{center}
			\vfill
			\large São Paulo \\
			\large\thedate
			\vspace*{1cm}
			\thispagestyle{empty}
	\end{center}

	\newpage
	
	%\widowpenalty10000
	%\clubpenalty10000
	\setlength{\parskip}{0.5cm}
	\setlength{\parindent}{1.1cm}
	\onehalfspacing	
	
	\section*{A questão}
	
	\setlength{\epigraphrule}{0pt}
%	\epigraph{\emph{
	\begin{epigraf}
		``É suficiente estabelecer que, se qualquer objeto de 
		aprendizado existe e o é apreensível pelo homem, é 
		preciso haver consenso entre quatro elementos ---
		o assunto ensinado, o professor, o aluno e o 
		método de ensino. Porém, como demonstraremos, não 
		existe o assunto, nem o professor, nem o aluno e 
		nem o método; portanto nenhum objeto de ensino existe.'' \\
		\\--- Sexto Empírico, \textit{Contra os Professores}%
	\end{epigraf}
	
	A filosofia é um pântano. Traiçoeira, difícil de se mapear, complexa 
	de se navegar. Além disso, sua própria existência é raiz de acalorados 
	debates. Logo, é fácil de se enxergar porque não desfruta a filosofia 
	dos mesmos benefícios que outras consagradas disciplinas. Há poucas 
	dúvidas sobre o que se ensinar em matemática. Ou mesmo em história, 
	ainda que haja, nesta última, espaço para amplos debates. 

	Outrora, a filosofia abrigara, sob sua guarida, simplesmente todo e 
	qualquer aspecto sob o qual se debruçava o pensamento humano, das 
	finas artes à medicina. Como se sabe, o progresso e a crescente 
	complexidade do pensamento humano tornou este desmembramento um 
	inevitável fato. Que papel, entretanto, restou à própria filosofia? 

	Se, noutro contexto, se perguntara \textit{o que é a filosofia?} 
	penso ser, aqui, particularmente conveniente que se indague 
	\textit{o que é a filosofia} hoje \textit{?} Afinal, é claro, 
	a compreensão das questões do ensino de filosofia depende 
	enormemente no entendimento acerca do que se espera da filosofia 
	nos atuais tempos. 

	O que se quer ouvir desta moribunda filosofia primeira, que 
	resiste tépida como um tímido comentário político nos jornais de 
	sábado ou meramente como uma coleção de vocábulos tão esvaziados 
	--- “biopoder”, “repressão”, dentre outros? Espera-se uma crítica? 
	Uma resposta às dúvidas existenciais clássicas? Palavras bonitas 
	para tomar inspiração antes de um dia caótico? 

	Leszek Kolakowski, arguto filósofo polonês, aponta: “Por bem mais 
	de cem anos, uma grande parte da filosofia acadêmica tem se 
	dedicado a explicar que a filosofia é impossível, inútil ou 
	ambas as coisas.” \footnote{\cite{lkmeta}, p. 13}
	E, ainda, que a despeito de tal tese ter sido 
	inúmeras vezes “provada”, “o adeus da filosolia é interminável, assim 
	como o \"bye-bye\" na famosa seqüência do filme de [o Gordo e o Magro].”
	\footnote{idem, ibid.}. De tal sorte que, se nada a filosofia 
	resta, ela ainda sobrevive a guisa de criticar a si própria. 

	O próprio vocábulo “filosofia” não existe em nosso usual vernáculo 
	sem imensos estigmas. Estigmas contra os quais, acredita-se, um 
	dos melhores métodos seja o \textit{rigor}, capaz de resgatar o que 
	seria uma “verdadeira” filosofia de dentro de um mar de espúrios 
	sofismos professados pelo “senso comum”. Mas será mesmo esta 
	filosofia, chancelada por um corpo acadêmico, aquela esperada 
	por nossos estudantes? É este o papel da filosofia? Um estático 
	passeio distante pela história do pensamento? 

	Por muitos anos, esta fora (e ainda é) a saída adotada por 
	muitos departamentos, dentre os quais aquele a que Paulo Arantes 
	chama de “Departamento Francês de Ultramar”. V. Goldschimidt 
	forjara, sobre a bigorna do rigor científico, este método, 
	que viria a ser chamado de estruturalismo. Acerca deste sistema, 
	evoquemos O. Porchat: 

	\begin{citac}
		Um primeiro passo indispensável e preliminar a toda análise 
		comparativa, a todo esforço de compreensão mais global, a 
		uma interpretação posterior mais geral de uma obra que 
		permita relacioná-la com seu contexto cultural, político, 
		econômico, e que propicie sua inserção numa perspectiva mais 
		propriamente histórica. \footnote{\cite{porchat}, p. 19} 
	\end{citac}

	Curiosamente, calha de ser adequadíssimo o exame desta questão sob 
	uma \textit{perspectiva histórica}. Do mesmo modo que a compreensão 
	da posição kantiana em a Crítica da Razão Pura depende um pouco do 
	entendimento acerca dos “dogmáticos” a quem Kant respondia, há, ao 
	estruturalismo golschmidtiano um contexto. E, a seu modo, buscava 
	responder, justamente, a pergunta que aqui se faz presente: o que 
	se espera da filosofia? 

	“O essencial de uma filosofia é uma certa estrutura”, nos lembraria 
	Goldschimidt, evocando Bréhier. \footnote{\cite{aradpfr}, p. 111}
	O uso da expressão é particular. Nos diz Paulo Arantes: 

	\begin{citac}
		[...] A lembrança é de Victor Goldschmidt, que, ao citar 
		a frase, não só reivindica o princípio como reconhece no 
		confronto que era a resolução da querela ideológica que 
		oporia anos mais tarde os partidários a história e os 
		adeptos da estrutura. Estes últimos certamente ignoravam 
		que Bréhier a compreendera como uma espécie de “digestão 
		espiritual, independentemente dos alimentos que seu tempo
		lhe propõe”. Uma ucronica (para falar ainda como Bréhier)
		que nos devolve ao coração da filosofia universitária [...]
		\footnote{Idem, ibid.}
	\end{citac}

	Não era a “estrutura”, entretanto, um conceito estranho. Tinha, 
	já no começo da contemporaneidade, fortes proponentes, tais como 
	Saussure, por exemplo. Sua aplicação a sistemas filosóficos 
	pré-data Goldschimidt, já existindo, à sua maneira, em nomes mais 
	antigos, tais como Lévi-Strauss, Maugüé e M. Gueroult. Impunha-se 
	à filosofia um rigoroso esforço arqueológico, procurando recompor 
	as obras tradicionais dentro de seus contextos. E não por menos: 
	não poucas vezes obras consagradas foram mal compreendidas e 
	demagogicamente empregadas com fins políticos, nesta nova era. 
	O rigor historiográfico parecia a única esperança de que se 
	pensar a filosofia sem cair em elucubrações espúrias.  

	Até aqui, o que se esperaria, neste sentido, da filosofia é apenas 
	que não traísse a si mesma, porquanto há, no aprendizado de história 
	da filosofia um rico instrumento de compreensão das estruturas que 
	nos permeariam. A prática filosófica que de tal esforço resulta 
	seria claramente muito mais organizada do que “uma tese descoberta, 
	que flutua livremente diante do espírito” \footnote{Idem, p. 114}. 

	O que Goldschimdt faz, basicamente, é dar a filosofia um 
	estruturalismo para chamar de seu, moldado às necessidades e 
	particularidades do texto filosófico. E sob sua supervisão 
	(e de discípulos como Porchat, Bento Prado Jr., J. A. Giannoti
	et. al) é que se deu a padronização do estudo filosófico. Tal 
	esforço fora importante não apenas para aplacar a sanha 
	existencialista, como, em terras brasileiras, serviu também 
	como única forma de se fazer filosofia durante um período de 
	intensa repressão política. 

	No entanto, assim como a esquisita tese do “filósofo engajado” 
	sartriano, é, de certo modo, uma reação a um certo \textit{zeitgeist}. 
	Como nos diz Porchat, não fora, naturalmente, em vão a adoção de tais 
	ideias, porquanto tiveram importante papel no melhoramento da formação 
	filosófia universitária. Contudo, ao mesmo tempo, escanteou nomes 
	como Lívio Teixeira e, ainda antes, Miguel Reale, o que afastou o 
	departamento do que se podia considerar mais próximo de uma filosofia 
	autóctone. E não apenas isto: criara uma cultura universitária em que 
	se desencorajaria a “reflexão própria”. 

	\begin{citac}
		Quero interrogar-me aqui, porém, sobre se essa é também a
		melhor maneira de preparar alguém para a prática da Filosofia, 
		para
		atender ao anseio original dos que vieram ao curso de Filosofia
		movidos por outra intenção que não a de tornar-se um dia bons
		historiadores do pensamento filosófico. Seus impulsos eram 
		filosóficos. Acredito que se pode dizer isso de um bom número 
		de nossos estudantes. E me ocorre, então, a seguinte pergunta, 
		que formularei com alguma brutalidade: estamos contribuindo 
		para a concretização desses impulsos, ou os estamos matando? 
		\footnote{\cite{porchat}, p.21}
	\end{citac}
	
	Logo, verifica-se estar a filosofia “acadêmica” também em maus lençóis. 
	Tivera uma contribuição crucial para a documentação e a discussão da 
	história das ideias. Entretanto, esta filosofia que, como aponta 
	Kolakowski, “sobrevive a sua própria destruição” ou esta que nos 
	entrevem durante solilóquios e pensamentos avulsos pode, de algum modo 
	sempre caber em “estruturas”, que lhe seriam “essenciais”? Se não 
	couberem, descartá-las, pura e simplesmente, será suficiente? 
	
	\newpage

	Pior: totalmente excluídas da universidade, ideias “semicultivadas”
	tendem a florescer fora dela, vendidas pela mais variada sorte de 
	charlatães que tomam populisticamente a palavra a título de discutir 
	“o que a universidade não quer discutir”. Ou seja, o exacerbado 
	estruturalismo pode até mesmo afastar a faculdade de filosofia das 
	discussões reais, que, no limite, seriam demasiado “toscas” para serem 
	debatidas seriamente. 

	As discussões no campo das humanidades nunca estiveram tão 
	fragmentadas. Nos telejornais a história é uma, nos perfis de Twitter 
	dos cientistas é outra e nos cafés filosóficos mais distantes ainda 
	outra. Portanto, nunca esteve tão difícil o acesso à “verdade”, esta 
	inexpugnável Deusa do filosófo. A cada recorte, apenas uma parte da 
	verdade se expõe. E este filósofo, em tempos de frenesi e informação 
	na velocidade da luz, navega águas muito caudalosas para poder se 
	ater a um fato de cada vez, com o devido cuidado. 

	O bom preceptor não pode ignorar esta conjuntura. Pensar um “curso” 
	de humanidades em meio a um momento de “crise das humanidades” pode 
	parecer uma loucura tão grande quanto alguém, por opção própria, 
	resolver cursar uma graduação em filosofia. Contudo, caso este 
	preceptor se atenha aos dizeres de Kolakowski, é possível divisar 
	uma saída: nada tão conveniente quanto um momento de 
	crise para se demonstrar as discussões fora das páginas amareladas 
	da filosofia de outrora. 

	O problema, todavia, vai muito além. Sexto Empírico, no trecho que 
	abre esta seção, não erra, ao apontar para “impossibilidade de se 
	ensinar”. E qualquer um que já tenha pisado numa sala de aula há de 
	entender o porquê. Não porque falte giz, “recursos multimídia” ou 
	material. Ao mesmo tempo que há o tempo e há o espaço, não há 
	\textit{comunidade} entre as partes envolvidas no processo. Há, 
	sobretudo, um problema de linguagem. 

	Afinal, assim como em Sexto, tantas verdades são possíveis que, 
	não raro, a aceitação de determinada visão de mundo há de 
	depender recursivamente de uma série interminável de referenciais 
	anteriores. Este obstáculo da linguagem, não obstante inexpugnável, 
	na pequena escala de uma sala de aula, há de ser o foco, de qualquer 
	modo, durante a construção de uma estratégia didática. 

	Está, enfim, circunscrita à feitura desta estratégia a pesquisa 
	pelos supostos “fins” da filosofia na atualidade, pois, ainda que 
	não haja para a questão que aqui se coloca -- “o que se espera, 
	hoje, da filosofia?” --, um bom mestre tem de ser capaz de 
	identificar, em seus púpilos, mesmo os mais tímidos impulsos 
	filosóficos, para o bem deles e para o bem da filosofia nos 
	tempos que se seguirão. 

	\newpage

	\subsection*{Dos percalços educacionais}

	A estes problemas linguísticos e epistemológicos que assombram 
	a disciplina, em si, se somam as mazelas sociológicas que abatem 
	sobre o ensino. É latente a descrença na educação e o mais completo 
	descrédito para com as ciências. No frigir dos ovos, a culpa é de 
	todos, e, também de ninguém. Não há, certamente, exagero algum em 
	afirmar que, se a educação não for a nós prioridade, também não 
	será ao estado. 

	Logo, é de pouca surpresa que encontremos escolas cujos ativos estão 
	em frangalhos e os quadros, incompletos. Ora, se não nos tem sido 
	sequer possível zelar pelos museus, estas casas famosas, responsáveis 
	por um patrimônio de inestimável valor, que esperança poderá ter 
	aquele mais afastado grupo escolar ou aquela mais distante escola 
	estadual? 

	De um ponto de vista material, é manifesta a péssima situação em 
	que se vê a escola pública. Em nada, entretanto, se compara à 
	situação sociológica: à guisa de combater uma educaçao contamida 
	pela ideologia, permite-se a nomeação de um ministro incapaz de 
	seguir as normas cultas da língua, chancela-se a recriação das 
	famigeradas “escolas especiais” e até se acena para a 
	possibilidade de se reativar manicômios. 

	Além disso, a massificação do ensino provocara um descompasso entre 
	as estratégias e o aluno (agora um público muito mais diverso), como 
	nos mostra J. N. Vicente, ao comentar as visões de M. Tozzi a respeito 
	do ensino de filosofia: ``
		[...] Os alunos que frequentam hoje o BAC [equivalente francês 
		do Ensino Médio] são linguística e culturalmente muito mais 
		deficitários que a elite das dećadas anteriores. ''
		\footnote{\cite{neves}, p. 399}

	

	\subsection*{Da possível saída}
	
	Não há de se negar: estamos e

	\newpage
	
	\section{Metodologia Geral}
	
	Em primeiro lugar, para se ministrar um curso, é preciso pensar na 
	estrutura de aulas, levando em conta tempo e espaço. Assumiremos que 
	as turmas serão constituidas por cerca de 40 alunos, com aulas de 
	duração padrão 50 minutos e baixos recursos. Dada a quantidade de 
	pessoas na sala, deveremos elaborar aulas que capturem a atenção dos 
	mesmos e construír atividades de curta (geralmente uma aula) duração.
	Também gostaríamos de relacionar as metas do parâmetro nacional com 
	conteúdos culturais como filmes e livros não didáticos.	

	\subsection*{Primeira Aula}

	O primeiro princípio a ser adotado aqui é que será usada a primeira aula 
	como uma espécie de ``termômetro'' da sala. Aplicaremos uma sorte de 
	ficha diagnóstica, na forma de uma atividade de sala com o intuito 
	de medir quanto a turma está familiarizada com os temas e linguagem 
	da filosofia. Esta ficha não será uma avaliação ou um teste dos 
	conhecimentos da turma sobre o conteúdo do programa, mas uma roda de 
	conversa para medir a disponibilidade dos alunos de participar de 
	discussões e pensar a filosofia. 
	
	O modelo `roda de conversa' foi eleito 
	para nos distanciarmos do estigma que acomete a disciplina de filosofia:
	que se trata de uma área árida e de pouca relevância prática.
	Uma possível forma de fazê-lo é evocar trechos e conceitos 
	clássicos da filosofia, como por exemplo aqueles encontrados em 
	Platão (\textit{Mito da Caverna}) e Kant (\textit{O que é 
	esclarecimento?}) e em seguida fomentar a discussão e interpretação das 
	passagens citadas.

	Deste debate espera-se ser possível ter ao menos alguma ideia da 
	``responsividade'' da turma para com certos conteúdos e ideias, 
	bem como o amadurecimento da mesma para participar das aulas 
	e discussões. Isto é que dará respaldo para a organização de 
	todo o conteúdo subsequente do curso. Será possível saber, dentre 
	outras coisas, qual o \emph{attention span} da turma, suas 
	pré-disposições a interagir com os temas propostos e a forma 
	como absorvem estes mesmos -- bem como seus preconceitos sobre 
	a disciplina que teremos que tratar no decorrer do curso. 
	
	É possível também fazer um exame das possíveis dificuldades 
	que serão encontradas neste percurso. Alguns exemplos de possíveis 
	percalços seguem:

	\begin{enumerate}[label=\alph*)]
		\item	\label{prob:engaj} 
			Os estudantes não se engajarem com a discussão 
			dos textos.

		\item	\label{prob:ridic}
			Os estudantes, em geral, não se abrirem para a 
			revelar o que pensam por medo de serem ridicularizados
			ou repreendidos.

		\item	\label{prob:disc}
			Discussões infrutíferas e falta de foco.
	\end{enumerate}
	
	Para estas, a sugerimos para \ref{prob:engaj} que a discussão
	seja direcionada para incluír os possíveis interesses de alunos
	-- mixturando coisas correntes com questões filosóficas: por
	exemplo, relacionando certos video-jogos e outras narrativas 
	mais acercadas aos alunos, com as questões subjacentes que os 
	mesmos carregam consigo. 
	
	No caso que ocorra \ref{prob:ridic}, pouco pode ser feito senão
	acolher os alunos na sua subjetividade e garantir que qualquer 
	um que se expresse não seja açoitado pelo ridículo de seus 
	colegas. Além disso, exemplos de interpretações e motivações 
	podem estimular a participação.

	Quanto a \ref{prob:disc}, a orientação do professor é sempre
	necessária. Discutir é, em geral, frutífero pois o propósito
	do ensino de filosofia para o ensino médio encompassa o 
	desenvolvimento do espírito crítico e do pensamento claro nos 
	alunos -- coisa esta que se constroi em ato e na fala, com os 
	outros, ou consigo.

	\subsection*{Segunda Aula}

	Em havendo estabelecido o timbre da sala, podemos seguir no 
	projeto de desmontar certas ascepções sobre a filosofia 
	enquanto produzimos vontade de conhecimento nos alunos. Nesta
	etapa, e segunda aula, a intenção é montar uma roda de conversa
	em um sentido mais amplo. No que em uma aula anterior foi uma 
	exposição mais descontraída de questões da filosofia, faremos 
	uma mudança ainda mais radical: desorganizar a ordem da sala e 
	mudar o foco da discussão para as perguntas dos alunos sobre 
	a própria vida. 

	Este movimento tem triplo propósito: I) Aproximar o 
	professor ao aluno fisicamente pela disposição dos 
	estudantes -- que idealmente se daria um ambiente aberto como 
	uma ágora; II) Subverter a ideia que o professor 
	derramará conhecimento que deve ser absorvido, como se houvesse
	um eixo central na sala e o conteúdo imanasse do mesmo. 
	Substituíndo-o pelo círculo, onde nenhum ponto é favorecido e 
	a compreensão advém de qualquer e toda parte. III) 
	Construir nos alunos um respeito para além da autoridade efetiva
	do professor em sua função designada, como será visto a seguir.

	A atividade em si é uma conversa livre -- ainda que orientada 
	pelo professor. O docente se disporia a responder perguntas dos 
	alunos acerca da vida em seu sentido mais amplo, prosaico e 
	ingênuo. E -- também -- fazer perguntas que não se espera que 
	os alunos saibam ou consigam responder, para que se estabeleça 
	de maneira evidente o objetivo do curso: que os alunos passem 
	também a serem capazes de, sem o recurso de livros e referências,
	formular respostas com qualidade de raciocínio significativa 
	para questões que eles se disponham a formular sobre suas 
	respectivas realidades.
	
	Estratégias para a mediação da conversa: Para garantir que todos 
	se envolvam na conversa de alguma maneira, o professor pode, uma 
	vez reunidos os alunos, iniciar com uma questão e eleger um aluno 
	para tentar respondê-lo. O aluno escolhido, por sua vez pode 
	perguntar e escolher um outro para o responder ou, ainda, ceder a 
	palavra a alguém que queira fazê-lo. Neste molde, o professor 
	ainda pode escolher os mais acanhados entre os pupilos mas sem 
	ser um centro da conversa pois sempre cabe ao locutor a 
	prerrogativa de escolher seu sucessor.
	
	Dos problemas que podem ser encontrados temos dois principais:
	a indisponibildade do espaço de discussão, que precisa ser calmo 
	para a formação de ideias, também, calmas e estruturadas e -- 
	também -- a indisposição dos alunos. No caso indisposição dos 
	alunos uma medida de insistência e indignação teatral devem 
	solucionar a pouca vontade de falar dos alunos haja visto que é 
	impossível que eles não tenham opinão sobre nenhum assunto ou 
	sejam incapazes de forma-la e, certamente, podem raciocinar sobre 
	o mundo pois são munidos de intelecto.

	Os próprios empecilhos encontrados podem servir de motivadores 
	para a conversa: se um aluno não permite que ela aconteça por 
	fazer barulho, há uma discussão que pode decorrer a respeito 
	de normas sociais, domesticação do corpo e mente, entre outros 
	tantos temas que uma mente aguçada aprende a encontrar nos 
	assuntos prosaicos da vida.

	\subsection*{Terceira e Quarta Aulas}
	
	Escolher-se-á um filme para exibir que tenha alguma relevância 
	para os temas das metas do programa. Durante o início da 
	exibição os alunos serão informados que deverão produzir um 
	trabalho de discussão de alguma questão que está presente no 
	filme, em formato livre, que de alguma forma elabore ou 
	explane ou contraponha questões ou problemáticas que julguem
	de interesse. 

	O objetivo desta atividade é duplo: por um, forçar os alunos 
	a praticarem o germe de pensamento autônomo que gostaríamos 
	de termos plantado nas aulas anteriores e, por outro, apresentar
	elementos de cultura geral de acesso mais restrito por 
	preconceitos (Barry Lyndon na lista de sugestões é um filme 
	que dura mais do que três horas) e desinteresse dos alunos para 
	que -- mesmo que os desinteresse -- conheçam.

	Entre o acervo que poderá ser exibido temos alguns filmes que 
	tem mérito de promover questões de ordem óbviamente filosófica:

	\paragraph{Ghost in the Shell} é uma animação 
		Japonesa década de 90 que aborda questões de 
		pessoalidade, existêncialismo, essência, 
		dualismo, transcendência, temas de política e 
		tecnologia no futurismo e distopia.

	\paragraph{O Show de Truman} que pode tratar tanto de 
		temas como vigilância, a natureza da realidade,
		absurdismo, alegoria da caverna.

	\paragraph{Barry Lyndon} que tem tanto a dizer sobre a 
		sociedade aristocrática, sobre a importância do
		acaso, do descontrole dos atores, falsa 
		civilidade, entre outros temas.

	\paragraph{Matrix} classicamente associado à alegoria 
		da caverna, poderia ser também interessante 
		fazer uma análise mais profunda das influências 
		das filosofias orientais que fundamentaram o 
		trabalho das Wachowskis. Elementos de Taoismo 
		poderiam adicionar um brilho a este tema já 
		muito conhecido.

	\paragraph{O Nome da Rosa} que rememora o tempo que 
		o conhecimento era intermediado e submetido ao 
		cerrado crivo da igreja, trata -- como queria 
		o seu autor -- de uma analogia a tempos modernos 
		mas igualmente tenebrosos que podem nos acometer.

	\paragraph{Akira} outra animação Japonesa de uma 
		sociedade em declínio e o perigo de se obter 
		poder incomensurável sem a capacidade e 
		maturidade de emprega-lo. Trata de juventude 
		marginalizada, contra cultura, jovens 
		delinquentes.

	\paragraph{Die Welle} ou ``A Onda''. O autoritativo 
		filme sobre a capacidade do ser humano de 
		produzir autoritarismo e fascismo, da banalidade
		da sua origem magnitude de seu desfecho.

	\paragraph{Chelovek s kino-apparatom} de Dziga Vertov e
		Elizaveta Svilova, filme metalinguístico com 
		elementos documentariais e experimentais. Obra 
		prima da década de 20.

	\paragraph{O Triúnfo da Vontade} Epítome da propaganda 
		Nazista, uma das obras máximas de Leni 
		Riefenstahl, excelente para produzir crítica e 
		repulsa nos alunos por ser um filme que busca 
		enaltecer -- pelos mesmos métodos que costumamos 
		empregar -- algo que fomos ensinados a abominar.
		
	Uma lista exaustiva é desnecessária e, sobretudo, 
	impossível de se elaborar em tempo hábil, mas não 
	ganharíamos muito com uma tal lista pois precisamos
	antes ler a sala para depois aplicar o filme.

	\subsection*{Quinta Aula}
	
	Interpretação do Filme --- Trabalho interpretativo do filme:
	Escolher um tema e/ou pergunta suscitada pela história ou 
	mídia e, a partir dela, desenvolver, na forma de uma redação, 
	em que o aluno desenvolve sua percepção sobre o tema. Trata-se 
	de uma atividade relativamente aberta. Que, entretanto, será 
	avaliada com um pouco mais de rigor, com vistas a garantir 
	que o discente ``mantenha o rumo'' da discussão. Ele terá de 
	tomar duplo cuidado para que sua redação não fuja nem à 
	discussão do filme e nem à do tema que esteja sendo tratado 
	no corrente momento. 

	\subsection*{Sexta e Sétima Aulas}
	
	Para o sexto encontro, organizar-se-á uma espécie de seminário. 
	O objetivo é garantir algum nível de ``engajamento'' por parte 
	do estudante. Este trabalho contempla diveras etapas, distribuídas 
	através de vários encontros. Para este primeira, espera-se que 
	os alunos se dividam em cerca de oito grupos (não muito mais do que 
	isso, por questões principalmente de tempo) e organizem-se para, 
	em cerca de dez minutos, fazer uma apresentação geral de algum
	filósofo. 
	
	Sugere-se que sejam apontados filósofos relativamente conhecidos 
	ao público de ensino médio. Alguns exemplos: Descartes, Hume, 
	Kant, Santo Agostinho, Platão, Nietzsche, Hegel e Locke. 
	
	\subsection*{Oitava e Nona Aulas}
	
	Os mesmos grupos preparam-se para, nesta aula, apresentar um 
	segundo seminário. Desta vez, entretanto, terão de ser mais 
	específicos. Escolherão uma obra do filósofo apresentado no 
	seminário anterior e farão, --- novamente --- em até dez minutos, 
	uma síntese desta obra e seus principais argumentos. 
	
	A título de exemplo, imagine-se que um grupo escolha apresentar, 
	digamos, um seminário acerca da \textit{Crítica da Razão Pura}, 
	de Kant. O grupo teria de explicar o contexto da obra (dogmáticos, 
	crise da razão, etc.), bem como algumas das soluções propostas 
	pelo pensador (doutrina transcendental dos conceitos etc.) Esta 
	apresentação não precisa ser avaliada com máximo rigor, mas quanto 
	mais próxima a descrição do grupo estiver da essência da obra, 
	tão melhor será o conceito atribuído ao final. 
	
	\subsection*{Décima e Décima Primeira Aulas}
	
	Aqui haverá a aplicação de um exame simples, contendo questões 
	de múltipla escolha, envolvendo conteúdos discutidos durante os 
	seminários. Alternativamente, pode-se pedir do aluno uma resenha 
	de dois ou três dos seminários. 
	
	Retorna-se, no encontro seguinte, aos alunos, todo o material 
	corrigido (ementas de seminário, resenhas, provinhas). Se possível, 
	procurar-se-á corrigir em sala o exame, que, quando for finalmente 
	devolvido ao estudante, deve estar idealmente comentado 
	pelo professor. 
	
	\subsection*{Décima Segunda e Décima Terceira Aulas}
	
	Torna-se aqui a assistir um filme. Pode ser um contido na 
	lista anterior ou não, mas tem de estar, na medida do possível, 
	alinhado com a temática principal que quer que esteja sendo 
	tratada no corrente momento. 
	
	Pode ser pedida alguma atividade acerca do filme. Mas, 
	levando-se em conta que, a esta altura, é bem provável que se esteja 
	já no fim do semestre e que o próximo encontro há de ser uma prova, 
	isto não parece conveniente nem para os alunos e nem para o professor, 
	que terá de gastar mais tempo corrindo estas atividades. 
	
	\subsection*{Décima Quarta e Décima Quinta Aulas}
	
	Avaliação em dois turnos. Nesta primeira aula do bloco, será aplicada 
	a prova final, contendo questões um pouco mais complicadas e 
	envolvendo todo o conteúdo até aqui tratado. A atividade é, 
	preferencialmente individual. 
	
	Na aula subsequente, os exames corrigidos (e, mais importantemente,  
	sem comentário algum do docente) são devolvidos aos seus alunos. Os 
	alunos deverão refazer as questões que erraram para que sejam 
    feitos os comentários finais. 
	
	\subsection*{Décima Sexta Aula}
	
	Em se tratando de ser uma aula de fim de semestre, após a avaliação, 
	pode ser uma oportunidade para melhorar o tratamento a alguns temas 
	discutidos, quer seja na forma de uma discussão ou de qualquer outra 
	atividade. 




	\newpage
	
	\section{Da avaliação}

	\section{Da bibliografia específica}

	\newpage

	\section{Considerações finais}

	
	\begin{thebibliography}{9}
		% bibtexar isto! 
		
		\bibitem{dubet} 
		BOURDIEU, Pierre. 
		\textit{A Escola Conservadora: as desigualdades 
		frente à escola e a cultura}. 
		In: \textit{Escritos de Educação} 
		Editora Vozes, Petrópolis, 2003.
				

	\end{thebibliography}
	
	
	
\end{document}
