\documentclass[12pt,a4paper]{article}
\usepackage[utf8]{inputenc}
\usepackage[portuguese]{babel}
\usepackage{amsmath}
\usepackage{amsfonts}
\usepackage{amssymb}
\usepackage{graphicx}
\usepackage{parskip}
\usepackage{setspace}
\usepackage{scrextend}
\usepackage{titling}
\usepackage{epigraph}
\usepackage{soul}
\usepackage{enumitem}
%\usepackage[activate={true,nocompatibility},final,tracking=true,kerning=true,spacing=true,factor=1100,stretch=10,shrink=10,spacing=nonfrench]{microtype}
\usepackage{xcolor}
\usepackage[left=3.00cm, right=2.00cm, top=3.00cm, bottom=2.00cm]{geometry}
\author{Pedro T. R. Pinheiro}
\date{2019}
\title{Um Plano de Curso}

%\usepackage{draftwatermark}
%\SetWatermarkText{Em progresso}
%\SetWatermarkScale{3}

\newenvironment{citac}{
	\begin{addmargin}[4cm]{1em} \footnotesize}{\normalfont \end{addmargin}
}

\newcommand{\subtitulo}{}
\newcommand{\disciplina}{EDF0424 - Metodologia do Ensino de Filosofia}
\newcommand{\departamento}{Departamento de Metodologia do Ensino}
\newcommand{\unidade}{FE - Faculdade de Educação}
\newcommand{\prof}{Paulo H. F. Silveira}

\begin{document}
	\begin{center}
				\textbf{
				\LARGE USP - UNIVERSIDADE DE SÃO PAULO \\
			}
			\Large \textsc{\unidade} \\
			\large \textsc{\departamento}\\
			\vspace*{1cm}
				
			Disciplina: \disciplina; \\Prof.: \prof
			\vfill
			\begin{center}
				{\Large \textsc{Pedro T. R. Pinheiro}} \\ 
				\vspace{1cm}
				\LARGE\textbf{\thetitle} \\
				\Large\emph{\subtitulo}
			\end{center}
			\vfill
			\large São Paulo \\
			\large\thedate
			\vspace*{1cm}
			\thispagestyle{empty}
	\end{center}

	\newpage
	
	%\widowpenalty10000
	%\clubpenalty10000
	\setlength{\parskip}{0.5cm}
	\setlength{\parindent}{1.1cm}
	\onehalfspacing	BOURDIEU
	
	\section*{Preâmbulo}
	
	%\setlength{\epigraphrule}{0pt}
	%\epigraph{\emph{``Como uma metrópole, o meu coração não pode parar / 
	% Mas também não pode sangrar eternamente''}}
	%{--- Belchior, \textit{Monólogo das Grandezas do Brasil}}
	
	A confecção de um curso para o ensino médio é menos trivial do que 
	pode, a priori, parecer. Da elaboração do material à produção do 
	espaço físico, é notável a necessidade de se levar em conta uma 
	série de elementos ambientais e sociais. E nem mesmo o mais completo
	estudo antropológico seria capaz de abarcar as mais plurais 
	configurações possíveis para uma dada turma. Logo, a feitura deste 
	dito plano de aula não é senão uma genérica descrição do que 
	parece a este autor de modo minimamente universal. É, por natureza, 
	um trabalho pouco prescritivo. 
	
	Dentre uma ampla gama de possíveis referências, opto por tentar, 
	na medida possível, encontrar o caminho que pareça ser mais 
	genérico e universal possível. Assim, servirá, de certo modo, mais como 
	um \textit{framework} do que como um curso em si. Pretende-se 
	elaborar um esqueleto onde podem se encaixar técnicas e modelos de 
	aula diversos. Claramente, haverá, uma certa necessidade de optar, 
	ao fim e ao cabo, por uma solução completa, mas há de ser preocupação 
	posterior. 
	
	Comumente, há, no jovem estudante do ensino básico, uma tendência a 
	voltar-se a assuntos de escopo bastante reduzido, frequentemente sob 
	a égide de seu próprio indivíduo. Nem todo o ser humano é, entretanto, igual, 
	como, por mesma lógica, nenhuma sala de ensino médio se porta de mesma 
	maneira que outra. Cada caso é um caso. Ora, não seria sensato pensar 
	o curso com base na receptividade dos indivíduos? 
	
	\textbf{TIRAR DO RABO UMA CITAÇÃO QUE CORROBORE ISSO}
	
	TERMINAR ESTA SECAO DEPOIS
	
	\newpage
	
	\section{Plano de Aulas}
	
	Em primeiro lugar, para se ministrar um curso, é preciso pensar na 
	estrutura de aulas, levando em conta tempo e espaço. Assumiremos que 
	as turmas serão constituidas por cerca de 40 alunos, com aulas de 
	duração padrão 50 minutos e baixos recursos. Dada a quantidade de 
	pessoas na sala, deveremos elaborar aulas que capturem a atenção dos 
	mesmos e construír atividades de curta (geralmente uma aula) duração.
	Também gostaríamos de relacionar as metas do parâmetro nacional com 
	conteúdos culturais como filmes e livros não didáticos.	

	\subsection*{Primeria Aula}

	O primeiro princípio a ser adotado aqui é que será usada a primeira aula 
	como uma espécie de ``termômetro'' da sala. Aplicaremos uma sorte de 
	ficha diagnóstica, na forma de uma atividade de sala com o intuito 
	de medir quanto a turma está familiarizada com os temas e linguagem 
	da filosofia. Esta ficha não será uma avaliação ou um teste dos 
	conhecimentos da turma sobre o conteúdo do programa, mas uma roda de 
	conversa para medir a disponibilidade dos alunos de participar de 
	discussões e pensar a filosofia. 
	
	O modelo `roda de conversa' foi eleito 
	para nos distanciarmos do estigma que acomete a disciplina de filosofia:
	que se trata de uma área árida e de pouca relevância prática.
	Uma possível forma de fazê-lo é evocar trechos e conceitos 
	clássicos da filosofia, como por exemplo aqueles encontrados em 
	Platão (\textit{Mito da Caverna}) e Kant (\textit{O que é 
	esclarecimento?}) e em seguida fomentar a discussão e interpretação das 
	passagens citadas.

	Deste debate espera-se ser possível ter ao menos alguma ideia da 
	``responsividade'' da turma para com certos conteúdos e ideias, 
	bem como o amadurecimento da mesma para participar das aulas 
	e discussões. Isto é que dará respaldo para a organização de 
	todo o conteúdo subsequente do curso. Será possível saber, dentre 
	outras coisas, qual o \emph{attention span} da turma, suas 
	pré-disposições a interagir com os temas propostos e a forma 
	como absorvem estes mesmos -- bem como seus preconceitos sobre 
	a disciplina que teremos que tratar no decorrer do curso. 
	
	É possível também fazer um exame das possíveis dificuldades 
	que serão encontradas neste percurso. Alguns exemplos de possíveis 
	percalços seguem:

	\begin{enumerate}[label=\alph*)]
		\item	\label{prob:engaj} 
			Os estudantes não se engajarem com a discussão 
			dos textos.

		\item	\label{prob:ridic}
			Os estudantes, em geral, não se abrirem para a 
			revelar o que pensam por medo de serem ridicularizados
			ou repreendidos.

		\item	\label{prob:disc}
			Discussões infrutíferas e falta de foco.
	\end{enumerate}
	
	Para estas, a sugerimos para \ref{prob:engaj} que a discussão
	seja direcionada para incluír os possíveis interesses de alunos
	-- mixturando coisas correntes com questões filosóficas: por
	exemplo, relacionando certos video-jogos e outras narrativas 
	mais acercadas aos alunos, com as questões subjacentes que os 
	mesmos carregam consigo. 
	
	No caso que ocorra \ref{prob:ridic}, pouco pode ser feito senão
	acolher os alunos na sua subjetividade e garantir que qualquer 
	um que se expresse não seja açoitado pelo ridículo de seus 
	colegas. Além disso, exemplos de interpretações e motivações 
	podem estimular a participação.

	Quanto a \ref{prob:disc}, a orientação do professor é sempre
	necessária. Discutir é, em geral, frutífero pois o propósito
	do ensino de filosofia para o ensino médio encompassa o 
	desenvolvimento do espírito crítico e do pensamento claro nos 
	alunos -- coisa esta que se constroi em ato e na fala, com os 
	outros, ou consigo.

	\subsection*{Segunda Aula}

	Em havendo estabelecido o timbre da sala, podemos seguir no 
	projeto de desmontar certas ascepções sobre a filosofia 
	enquanto produzimos vontade de conhecimento nos alunos. Nesta
	etapa, e segunda aula, a intenção é montar uma roda de conversa
	em um sentido mais amplo. No que em uma aula anterior foi uma 
	exposição mais descontraída de questões da filosofia, faremos 
	uma mudança ainda mais radical: desorganizar a ordem da sala e 
	mudar o foco da discussão para as perguntas dos alunos sobre 
	a própria vida. 

	Este movimento tem triplo propósito: I) Aproximar o 
	professor do aluno tanto fisicamente pela disposição dos 
	estudantes -- que idealmente se daria um ambiente aberto como 
	uma ágora; II) Subverter a ideia que o professor 
	derramará conhecimento que deve ser absorvido, como se houvesse
	um eixo central na sala e o conteúdo imanasse do mesmo. 
	Substituíndo-o pelo círculo, onde nenhum ponto é favorecido e 
	a compreensão advém de qualquer e toda parte. III) 
	Construír nos alunos um respeito para além da autoridade efetiva
	do professor em sua função. Este último ponto será explanado 
	após se esclarecer o que propomos para este dia da aula.

	A atividade em si é uma conversa livre -- ainda que orientada 
	pelo professor. O docente de disporia a responder perguntas dos 
	alunos acerca da vida em seu sentido mais amplo, prosaico e 
	ingênuo. E -- também -- fazer perguntas que não se espera que 
	os alunos saibam ou consigam responder, para que se estabeleça 
	de maneira evidente o objetivo do curso: que os alunos passem 
	também a serem capazes de sem o recurso de livros e referências
	formular respostas e com qualidade de raciocínio significativas 
	para questões que eles se disponham a formular sobre suas 
	respectivas realidades.
	
	Estratégias para a mediação da conversa; Possiveis problemas, 
	aluno \emph{fdp} etc

	\subsection*{Terceira Aula}
	Filme -- Escolher filme, explicar relevância da escolha e do 
	conteúdo per si. Possíveis filmes: fodaçe

	\subsection*{Quarta Aula}
	Interpretação do Filme --- Trabalho interpretativo do filme:
	Escolher um tema e/ou pergunta sussitada pela história ou 
	mídia (por exemplo, digamos, ``falar da luz nos filmes do 
	Kubrick'' ou sei lá ``Macbeth did nothing wrong'')

	Seu formato vai ser livre, mas a avaliação vai ser nabo pq
	se é livre tem que ser mais foda --- ser vira negada.	

	\section{Termina, Porra Rodrigues Tadeu Pinheiro}
	{\Huge\LaTeX}


	\newpage
	
	\begin{thebibliography}{9}
		% bibtexar isto! 
		
		\bibitem{brd} 
		BOURDIEU, Pierre. 
		\textit{A Escola Conservadora: as desigualdades frente à escola e a cultura}. 
		In: \textit{Escritos de Educação} 
		Editora Vozes, Petrópolis, 2003.
		
	\end{thebibliography}
	
	
	
\end{document}
