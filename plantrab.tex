\documentclass[12pt,a4paper]{article}
\usepackage[utf8]{inputenc}
\usepackage[portuguese]{babel}
\usepackage{amsmath}
\usepackage{amsfonts}
\usepackage{amssymb}
\usepackage{graphicx}
\usepackage{parskip}
\usepackage{setspace}
\usepackage{scrextend}
\usepackage{titling}
\usepackage{epigraph}
%\usepackage[activate={true,nocompatibility},final,tracking=true,kerning=true,spacing=true,factor=1100,stretch=10,shrink=10,spacing=nonfrench]{microtype}
\usepackage{xcolor}
\usepackage[left=3.00cm, right=2.00cm, top=3.00cm, bottom=2.00cm]{geometry}
\author{Pedro T. R. Pinheiro}
\date{2019}
\title{Um Plano de Curso}

%\usepackage{draftwatermark}
%\SetWatermarkText{Em progresso}
%\SetWatermarkScale{3}

\newenvironment{citac}{\begin{addmargin}[4cm]{1em} \footnotesize}{\normalfont \end{addmargin}}

\newcommand{\subtitulo}{}
\newcommand{\disciplina}{EDF0424 - Metodologia do Ensino de Filosofia}
\newcommand{\departamento}{Departamento de Metodologia do Ensino}
\newcommand{\unidade}{FE - Faculdade de Educação}
\newcommand{\prof}{Paulo H. F. Silveira}

\begin{document}
	\begin{center}
				\textbf{
				\LARGE USP - UNIVERSIDADE DE SÃO PAULO \\
			}
			\Large \textsc{\unidade} \\
			\large \textsc{\departamento}\\
			\vspace*{1cm}
				
			Disciplina: \disciplina; \\Prof.: \prof
			\vfill
			\begin{center}
				{\Large \textsc{Pedro T. R. Pinheiro}} \\ 
				\vspace{1cm}
				\LARGE\textbf{\thetitle} \\
				\Large\emph{\subtitulo}
			\end{center}
			\vfill
			\large São Paulo \\
			\large\thedate
			\vspace*{1cm}
			\thispagestyle{empty}
	\end{center}

	\newpage
	
	%\widowpenalty10000
	%\clubpenalty10000
	\setlength{\parskip}{0.5cm}
	\setlength{\parindent}{1.1cm}
	\onehalfspacing
	
	\section{Preâmbulo}
	
	%\setlength{\epigraphrule}{0pt}
	%\epigraph{\emph{``Como uma metrópole, o meu coração não pode parar / Mas também não pode sangrar eternamente''}}{--- Belchior, \textit{Monólogo das Grandezas do Brasil}}
	
	A confecção de um curso para o ensino médio é menos trivial do que 
	pode, a priori, parecer. Da elaboração do material à produção do 
	espaço físico, é notável................ etc. 

	\newpage
	
	
	\begin{thebibliography}{9}
		% bibtexar isto! 
		
		\bibitem{brd} 
		BOURDIEU, Pierre. 
		\textit{A Escola Conservadora: as desigualdades frente à escola e a cultura}. 
		In: \textit{Escritos de Educação} 
		Editora Vozes, Petrópolis, 2003.
		
	\end{thebibliography}
	
	
	
\end{document}
