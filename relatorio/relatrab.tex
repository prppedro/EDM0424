\documentclass[12pt,a4paper]{article}
\usepackage[utf8]{inputenc}
\usepackage[portuguese]{babel}
\usepackage{amsmath}
\usepackage{amsfonts}
\usepackage{amssymb}
\usepackage{graphicx}
\usepackage{parskip}
\usepackage{setspace}
\usepackage{scrextend}
\usepackage{titling}
\usepackage{xcolor}
\usepackage{epigraph}
\usepackage{abntcite}
%\usepackage{draftwatermark}
%\SetWatermarkText{Em progresso}
%\SetWatermarkScale{3}

\newenvironment{citac}{\begin{addmargin}[4cm]{1em} \footnotesize}{\normalfont \end{addmargin}}

\usepackage[left=3.00cm, right=2.00cm, top=3.00cm, bottom=2.00cm]{geometry}
\author{Pedro T. R. Pinheiro}
\date{São Paulo\\2020}
\title{Relatório de Estágio}

\newcommand{\subtitulo}{A filosofia por ensino remoto}
\newcommand{\disciplina}{EDF0424 - Metodologia do Ensino de Filosofia II}
\newcommand{\departamento}{Departamento de Metodologia do Ensino}
\newcommand{\unidade}{FE - Faculdade de Educação}
\newcommand{\prof}{Paulo H. F. Silveira}

\begin{document}
	\begin{center}
				\textbf{
				\LARGE USP - UNIVERSIDADE DE SÃO PAULO \\
			}
			\Large \textsc{\unidade} \\
			\large \textsc{\departamento}\\
			\vspace*{1cm}
				
			Disciplina: \disciplina; \\Prof.: \prof
			\vfill
			\begin{center}
				{\Large \textsc{\theauthor}} \\ 
				\vspace{1cm}
				\LARGE\textbf{\thetitle} \\
				\Large\emph{\subtitulo}
			\end{center}
			\vfill
			\large\thedate
			\vspace*{1cm}
			\thispagestyle{empty}			
	\end{center}

	\newpage

	\widowpenalty10000
	\clubpenalty10000
	\setlength{\parskip}{0.5cm}
	\setlength{\parindent}{1.1cm}
	\onehalfspacing
	
	\section{Preâmbulo}

	\subsection{Sobre este ensaio e seu escopo}

	A discussão presente neste ensaio engloba as experiências de ensino 
	à distância proporcionadas pelo Centro de Mídias do Estado de 
	São Paulo (CMSP) e da Secretaria da Educação do Estado do Paraná 
	(SEED). No primeiro caso, fora acompanhado o docente Marconi F. L. 
	Cabral, M. F. Degan e L. Gonçalves, enquanto, no segundo, 
	acompanhara-se 	R. Polesi e L. H. V. Silva. Como sempre, o foco é 
	fazer uma análise da estrutura do curso, como os materiais e recursos 
	disponíveis são utilizados e, também, a resposta das turmas envolvidas 
	às estratégias 	correntemente adotadas. 
	
	Há de se frisar o imenso abismo entre as discussões realizadas no curso 
	e a dita ``realidade'': se por um lado temos, dentro dos muros da 
	universidade, a possibilidade de discutir métodos, ali, na prática, 
	defronte a uma sala de aula, prevalece o que coloquialmente poder-se-ia 
	rotular de ``vai como dá''. Como é de se esperar, a parte final deste 
	relatório procura, dentro da medida do possível, fechar este abismo. 
	
	Fechar este abismo -- o que, em si, já constitui numa tarefa hercúlea --
	é, contudo, meramente um primeiro passo. Com isto é 
	apenas possível diagnosticar algumas mazelas. E é ainda maior desafio 
	propor soluções. Não se desvela, ao jovem professor, de pronto, 
	a vasta gama de ferramentas para se trabalhar com os mais diversos 
	tipos de classes. Pelo contrário, são, inicialmente, muito limitados 
	os recursos com que conta o licenciado neófito. Deste modo, seria 
	demasiado ambicioso esperar deste relatório uma solução efetiva 
	às imperfeições observadas. 

	Logo, é apenas isto: uma tentativa. Tentativa de compreensão e 
	tentativa de solução. Afinal, sob certo ponto de vista, por toda a 
	volatilidade encontrada, e por toda a sua incerteza relacionada a 
	ser uma ciência que lida diretamente com os nervos da natureza 
	humana, não será ela inevitavelmente e quase exclusivamente empírica? 
	
	\newpage
	
	\section{Materiais e Organização do Curso}

	Em razão da crise sanitária, fora adotado, em ambos os casos, o 
	modelo de aula remota, com todos os seus percalços e limitações, 
	que serão discutidas mais adiante. A plataforma escolhida fora o 
	YouTube, serviço fundado em 2005 e comprado pela Google poucos 
	anos depois, consolidado como a mais importante plataforma de 
	streaming gratuita da internet. A CMSP ministrou seu curso, 
	primariamente, de maneira síncrona (isto é, ao vivo), ao passo de 
	que a SEED o fez de maneira assíncrona. 

	De um modo geral, a proposta principal não difere muito entre 
	os dois oferecimentos. Em ambos os casos, os vídeos servem, ao que 
	parece, de material de apoio aos professores das redes. Materiais 
	e atividades são fornecidos, em ambos os casos, por dentro de 
	plataformas e aplicativos oferecidos pelas instituições, pelos 
	professores responsáveis por cada turma. Logo, os encontros, 
	em si, constituem, de certa forma, em parte do material que 
	está sendo oferecido aos estudantes. 

	Há, não obstante, diferenças relevantes em abordagem, forma e 
	dinâmica dos conteúdos, entre as duas \textit{approachs}. A 
	metodologia adotada pelo CMSP parece dar considerável ênfase 
	à participação dos estudantes, incentivando-os a participar das 
	discussões interativamente, a partir de um chat disponível dentro 
	do aplicativo do Centro de Mídia. Ao passo de que a SEED parece 
	aderir a um modelo expositivo mais tradicional, em certa medida 
	mais conteudista e focado em tópicos de história da filosofia. 
			
    	Uma outra diferença entre o curso oferecido pela SEED e o curso 
    	da CMSP é que o primeiro costuma pedir atividades pela plataforma 
    	interna. Estas atividades constituem, também, numa forma de controle 
    	presencial. Infelizmente, não é possível a ouvintes ter fácil acesso 
    	a estas atividades, motivo pelo qual não serão, obviamente, aqui 
    	comentadas. 

	Chama, particularmente, no caso da SEED, a atenção os tópicos 
	relativamente avançados tratados nos módulos voltados para o 
	terceiro ano. Existe, na plano da SEED, uma clara intenção de 
	informar com relativa riqueza de detalhes cada corrente filosófica 
	e seu contexto histórico, ao passo de que as aulas do CMSP tendem 
	a fazer sobrevôos sobre os tópicos, delegando aos alunos a tarefa 
	de se informarem com mais profundidade. 

	Esta diferença estratégica parece sugerir que há, por parte do 
	CMSP, uma tendência de procurar centralizar os pontos de debate 
	conforme os currículos, mas que, no fim das contas, é imprescindível 
	que cada professor oriente adequadamente sua turma para além do 
	escopo dos oferecimentos online.   

	\newpage
	
	\section{As aulas}

	\subsection{SEED - 1º ano E.M. - 6 de agosto}

	\textbf{Tema:} Copérnico e o Heliocentrismo

	Ministrada no começo de agosto, é a trigésima-quarta aula do curso. 
	Neste encontro, Polesi faz uma exposição histórica do tema. Seu 
	ponto de partida é contexto histórico, i.e. quem fora Copérnico, 
	qual fora sua contribuição para o pensamento científico. Em seguida, 
	em um slide, é proposta a seguinte questão: “O que significa revolução 
	copernicana?” 

	O professor confere aos alunos ao menos cinco minutos para tentar 
	formular uma resposta, antes de apresentar sua própria resposta à 
	questão, no slide seguinte. Daí em diante, procede com a explicação 
	acerca do método copernicano. Faz também uma exposição das influências 
	intelectuais de copérnico, e, para garantir que os alunos lembrariam 
	do conteúdo, propõe uma questão para ajudá-los nas anotações: “Qual 
	foram as influências de Copérnico em seu pensamento?”, à qual precisam 
	responder em cinco minutos. 

	Em seguida, apresenta, novamente, a resposta correta, pontificando as 
	ideias principais da seção (influências realistas e neoplatônicas). 
	Sem mais delongas, prossegue, então, continua em direção ao próximo 
	assunto, explorando mais profundamente a situação da astronomia 
	antes de Copérnico e como Copérnico instaura uma quebra paradigmática 
	em relação à colcha de retalhos que era a teoria astronômica outrora. 

	Encerrando este passo, há uma outra questão: “Por que as respostas da 
	teoria ptolomaica deveriam ser superadas?”, convidando, logo, os 
	estudantes a recapitular os motivos pelos quais Copérnico rompera com 
	a “tradição” nos estudos em astronomia. Cinco minutos depois, Polesi 
	retorna, comentando sobre a ausência à dita teoria ptolomaica de um 
	“corpo fundante”, elemento crucial a uma teoria científica. 

	Neste momento, se aproxima ao final da aula, e o docente avança à 
	análise das consequências do pensamento copernicano e de seu legado 
	para a modernidade, tendo influenciado Newton, Descartes, Kepler 
	(que continuou seu trabalho) e outros. Para sumariar, é posta, então, 
	a última pergunta do encontro: “qual foi a principal contribuição de 
	Copérnico?“, o que parece convidar os alunos a revisitar com mais 
	profundidade todo o conteúdo exposto neste dia. 

	Polesi encerra o encontro lendo rapidamente a resposta para a questão 
	proposta e expõe alguns slides contento referências bibliográficas 
	interessantes aos interessados em se aprofundar no assunto. 

	\subsection{CMSP - 1º ano E.M. - 13 de agosto}

	\textbf{Tema:} As Formas de Governo em Montesquieu

	Ministrado por M. Cabral e mediado por M. Degan, este encontro começa, 
	após uma breve introdução, com uma indagação: “Em sua visão, qual a 
	melhor forma de governo? Por quê?“ Os alunos respondem no chat, 
	ao longo de um minuto e M. Degan comenta brevemente com Cabral algumas 
	das respostas antes de entrar no tópico teórico, que se inicia com 
	um slide de recapitulação dos assuntos anteriores apresentado por 
	Cabral. 

	Como foram abordados diversos tópicos de filosofia política no 
	encontro antrior e, também, fora sugerida uma pesquisa acerca de 
	Montesquieu e a ideia dos Três Poderes, este autor surge quase 
	como natural ponto de partida do presente encontro. Degan retoma 
	o autor, expondo de forma sucinta sua vida, sua obra e suas 
	principais ideias (lei, divisão dos poderes etc.) 

	O próximo passo é, então, elencar as formas de governo que serão 
	discutidas. Neste primeiro momento, fala-se sobra monarquia e 
	república, as duas primeiras formas descritas por Montesquieu. 
	Cabral pontua quais os princípios fundamentais e 
	características principais de cada forma. Findada esta passagem, 
	é proposto um exercício, contendo quatro afirmações sobre os 
	tópicos até aqui tratados. Dá-se, então, aos estudantes, dois 
	minutos para indentificar a veracidade ou não de cada afirmação. 

	O docente corrige o exercício, e continua para a terceira forma: 
	o governo despótico. Descreve o conceito e a formato deste tipo 
	de governo e correlaciona a descrição de Montesquieu com a teoria 
	de Maquiavel. Em seguida, aponta o caráter anti-político desta 
	forma de governo e aponta possíveis maneiras de se evitar o 
	despotismo, de acordo com a solução montesquiana. 

	E, após conectar este tópico com a ideia de democracia e cidadania, 
	Cabral coloca ainda outra atividade, pedindo para que os estudantes 
	respondam, no chat, a seguinte questão: “Pensando na ideia de 
	cidadania, a quem cabe fiscalizar a atuação dos governantes?” 

	Trinta segundos depois, Cabral comenta sobre cidadania e papel 
	do cidadão, brevemente. E, como já se aproxima o final da aula, 
	é posta apenas uma última questão acerca da fiscalização do 
	governante por parte do cidadão. Esta questão, entretanto, fica 
	em aberto para um próximo encontro. 

	Cabral resume os pontos tratados no encontro, faz alguns anúncios e, 
	sugere alguns conceitos a serem pesquisados pelos alunos e, 
	após as considerações finais de Degan, encerra-se o encontro. 

	\subsection{SEED - 3º ano E.M. - 2 de setembro}

	\textbf{Tema:} John Dewey -- Funcionalismo e Pragmatismo

	L. Silva, após típicos lembretes e recados, inicia a aula 
	fazendo uma apresentação contextual do chamado pragmatismo, 
	principal corrente de pensamento americana. Em seguida, 
	apresenta com profundidade o extenso currículo de Dewey e 
	o impacto de suas ideias na filosofia e na educação. Frisa 
	o importante papel de Dewey à fundação do movimento pela 
	escola progressiva e da doutrina pragmática. 

	Estranhamente, a primeira questão da aula é um exercício 
	de verdadeiro/falso que indaga pela validade da definição 
	de pragmatismo dada pelo professor poucos minutos antes. 
	Cinco minutos são dados para se assinalar o valor de verdade 
	da afirmação. 

	Silva corrige a questão, e prossegue para uma detalhada 
	explicação exposição dos preceitos básicos do pensamento 
	pragmático a partir de William James e o próprio Dewey. 
	Chama, também, a atenção para pontos nevrálgicos 
	da teoria pragmática. 

	A próxima questão indaga simplesmente pela proposta do 
	pragmatismo, convidando os alunos a condensarem em poucas 
	linhas os vários pontos apresentados por Silva. 

	Corrige-se, então, brevemente a questão, findados os típicos 
	cinco minutos. E, a seguir, o docente passa a explicar a 
	filosofia de Dewey, a este ponto chamada “instrumentalismo” 
	pelo próprio autor, e como ela culmina no que se classifica, 
	posteriormente, como uma postura funcionalista. 

	Ao cabo desta explicação, indaga-se se a seguinte afirmação 
	é verdadeira ou falsa: “Para Dewey, a experiência social, em 
	detrimento de princípios absolutos, é ncessária para avaliar 
	o valor de uma ideia ou prática, o afastando de qualquer 
	funcionalismo”. E, sete minutos depois, Silva retorna, 
	explicando brevemente o porquê de ser falsa. 

	O docente finaliza o encontro resolvendo uma questão típica 
	de vestibular envolvendo correntes filosóficas tratadas pelo 
	curso até o momento e apresentando um resumo dos pontos 
	mais relevantes. 

	\subsection{CMSP - 2º ano E.M. - 10 de setembro}

	\textbf{Tema:} Filosofia e Gênero

	Logo de início, na conferência, M. Degan exibe um slide contendo 
	informações gerais sobre o tópico e algumas sugestões de pesquisa 
	que, posteriormente, podem vir a calhar dentro da discussão sobre 
	o tema. Em seguida, M. Cabral complementa as sugestões do slide 
	com mais materiais e aponta resumidamente a problemática do tópico. 
	Então, como já observado noutros encontros, é posta uma questão: 
	“Quais semelhanças e diferenças você percebe entre mulheres e 
	homens, atualmente?”

	Degan comenta brevemente algumas respostas e passa a palavra a 
	Cabral, que disserta sobre os conceitos de discriminação, 
	desigualdade e preconceito, temas que norteiam o presente 
	encontro, que estabelecerá o solo para o tema da crítica, a 
	ser tratado noutro momento. 

	Prossegue-se então a uma recapitulação histórica do problema do 
	gênero, iniciando-se por um recorte do pensamento iluminista em 
	Kant e Rousseau, contrastando o papel da mulher na esfera 
	privada ao papel do homem na esfera pública e como, efetivamente, 
	isto restringiu a possibilidade de crescimento intelectual 
	feminino. 

	Interessantemente, Degan aponta que os alunos participantes 
	atacaram passionalmente a Rousseau. E aproveita o momento para 
	observar que existe um contexto próprio a cada autor e que 
	julgar estas ideias com um olhar atual incorreria em profundo 
	anacronismo. Cabral complementa comentando que, justamente, 
	era o pensamento iluminista uma novidade e, por isso, ele 
	próprio não estaria livre de certas amarras da tradição que 
	postulava criticar. 

	Após uma rápida discussão sobre as funções de gênero de acordo 
	com a tradição vs. o gênero sob uma perspectiva de emancipação, 
	Cabral relembra os pontos principais da aula, reforçando a 
	recomendação de que pesquisem os pontos mencionados ao início da 
	conferência. E, então, coloca-se outra questão: “quais consequências 
	práticas as semelhanças e diferenças entre homens e mulheres 
	provocam em nossa sociedade?” 

	São expostas, a seguir, algumas possibilidades que se desvelaram 
	ao gênero feminino. Apesar disso, Marconi pontua uma série de 
	mazelas que ainda causam desigualdade entre gêneros e que, no 
	limite, acabam inclusive por afetar ambos os gêneros. 

	Já ao fim da exposição, é proposto um último exercício de escolha 
	entre afirmações verdadeiras e falsas, englobando todos os tópicos 
	tratados durante o encontro. Em razão do tempo escasso, Cabral 
	resolve o exercício, faz algumas considerações finais e, em seguida, 
	é encerrada a conferência, com algumas indicações de material para 
	professores e alunos da rede. 

	\subsection{SEED - 3º ano E.M. - 21 de outubro}

	\textbf{Tema:} Hegel -- O Belo na Arte

	Esta exposição consiste numa exploração da teoria hegeliana da 
	estética, disciplina que vinha sendo discutida por L. Silva já 
	há alguns encontros. O docente começa por relacionar o assunto 
	da aula ao contexto do módulo e, em seguida, faz, como já 
	observado na aula sobre Dewey e Pragmatismo, uma apresentação 
	relativamente detalhada do pensador em questão: Hegel. 

	Após conectá-lo com o contexto do idealismo alemão e a reação 
	à teoria kantiana, Silva envereda pelos complexos meandros do 
	sistema hegeliano, procurando explicar os conceitos empregados 
	pelo filósofo da forma mais clara possível. Discute resumidamente 
	o conceito de espírito na metafísica hegeliana, ponto de particular 
	importância na teoria do autor. E, logo após, dispõe, num slide, 
	uma lista das cinco principais obras hegelianas. 

	A atividade que se pede em seguida é simples: identificar o 
	movimento filosófico a qual Hegel faz parte. Passados quatro 
	minutos, o professor dá a resposta a correta e, sem mais delongas, 
	avança em direção ao próximo tópico: a dialética hegeliana, aqui 
	apresentada em seu modelo mais simplificado. 

	Silva explica que, enquanto o pensamento dialético não é, no 
	tempo de Hegel, novidade alguma, o filósofo tem uma particular 
	interpretação do que seria a dialética, que, embora complexa -- 
	Silva alerta que universidades têm cursos extensos tratando 
	apenas deste singelo ponto -- pode ser resumida em seu modelo 
	mais consagrado de tese/antítese e síntese, exposto por Silva 
	ao longo dos próximos minutos. 

	E, por fim, não deixa de mencionar um importantíssimo aspecto 
	da dialética hegeliana, que é sua má compreensão e sua presença 
	nos mais diversos nichos do leque ideológico resultantas das 
	frequentes apropriações da ideia sem a devida compreensão de 
	seu contexto no sistema hegeliano. 

	A segunda atividade pede, então, dos alunos, um resumo do 
	funcionamento da dialética hegeliana. Silva corrige rapidamente 
	o exercício e faz, finalmente, uma breve explanação sobre a 
	relação entre arte e natureza na estética hegeliana. 

	Neste ponto, coloca-se uma terceira questão, com o objetivo 
	de solidificar um importante ponto: o porquê de Hegel 
	modificar esta relação -- entre arte a natureza. 

	A aula, finalmente, é arrematada pela correção desta questão 
	e da resolução de um exercício de vestibular, como outrora 
	também observado. 

	\subsection{CMSP - 3º ano E.M. - 26 de novembro}

	\textbf{Tema:} Pensamento e Linguagem

	Esta conferência é apresentada por M. F. Degan, com mediação 
	de M. Cabral. E seu principal intuito é apresentar a forma 
	como a filosofia observa a questão da linguagem. De início, 
	Degan propõe a seguinte questão: “Como você definiria a palavra 
	`linguagem'?” 

	Após um breve comentário de Cabral acerca das respostas recebidas, 
	a professora prossegue, expondo como as questões acerca da 
	linguagem culminaram na chamada Filosofia da Linguagem. Degan 
	expõe num slide -- e comenta -- as principais questões desta 
	corrente filosófica. 

	Conecta, também, a questão da linguagem aos primórdios do pensamento 
	científico moderno -- encontrando-a dentre os sete enigmas de Emil 
	du Bois-Reymond, ainda no século XIX --, bem como na filosofia 
	do período clássico, ainda em Platão. 

	Degan prossegue, expondo a abordagem rousseauniana em “Ensaio sobre a 
	Origem das Línguas”, e como aspectos de linguagem influenciam na 
	produção do conhecimento e das artes. A docente também apresenta 
	a percepção artística da questão, enquanto comenta acerca do caráter 
	convencional da linguagem, que é, sobretudo, fruto de acordo e 
	construção coletiva de um código, inerente a um local, um grupo, 
	uma sociedade. 

	A seguir, revisita a concepção schopenhaueriana da palavra como 
	elemento “eternizado” pela pena do poeta, do escritor, do filósofo. 
	Pontua que a palavra bem empregada conserva para o futuro um 
	sentimento, um pensamento, e como isto é de vital importância 
	para o desenvolvimento do pensamento humano. 

	A professora ainda compara a tese schopenhaeuriana com a tese 
	rousseauniana, explicando que, no primeiro, a linguagem tem 
	a resiliência como seu principal elemento, ao passo de que, no 
	último, é basilar à linguagem ser fruto de uma convenção. 

	Para arrematar o ponto, Degan apresenta algumas sugestões de leitura 
	que englobam sites, e livros -- como Farenheit 451, entusiasticamente 
	indicado pela docente. 

	A exposição se encerra com uma pergunta acerca da facilidade de 
	apreensão dos conteúdos e um rápido comentário de Cabral acerca 
	dos comentários submetidos por outros professores. 
	
	\newpage
		
	\section{Análise dos relatos}

	Recapitulando o que fora comentado na seção II, lembremo-nos de que 
	estão em discussão, aqui, duas diferentes propostas de curso. Antes de 
	se ater ao exame de cada uma delas, parece uma ideia sensata definir, 
	aqui, um certo norte. O que se espera de um estudante de filosofia 
	no Brasil, pragmaticamente falando? A esta questão, busquemos uma 
	resposta nos parâmetros curriculares nacionais: 

	\begin{citac}
		[...]
		o que se quer enfocar é a necessidade de desenvolver no aluno 
		um  olhar especificamente filosófico, vale dizer, analítico,  
		investigativo, questionador, reflexivo, que possa contribuir 
		para uma compreensão mais profunda da produção textual 
		específica  que  tem sob as vistas. 
		\footnote{\cite{pcn}, p. 54}
	\end{citac}

	Os caminhos indicados pelo PCN apontam, portanto, na direção de um 
	curso de filosofia que não apenas informa os alunos historicamente, 
	mas os coloca em condição de debate, procurando dotá-los não apenas 
	de fatos históricos, mas também de instrumentação para que se elabore 
	uma crítica, uma observação. Ainda é dito, mais à frente: 

	\begin{citac}
		[...]
		o desenvolvimento dessa competência tem implicações que extrapolam o alcance de  um  curso  de  Filosofia  meramente  disciplinar.  Seria  preciso  ir  além  disso  e  trazer  para  a  prática cotidiana do aprender a filosofar (na medida do possível) alguns casos exemplares de outros  textos,  em  diferentes  suportes,  que  não  o  texto  especificamente  filosófico.
		\footnote{Idem, ibid.}
	\end{citac}

	Deste modo, é possível estabelecer que, de acordo com os parâmetos, 
	é esperado que o aluno seja ensinado, de certa forma, a ”pensar”. 
	Levar a cabo tal empreitada, reconhece-se, é difícil
	\footnote{
	“Considerando  o  critério  da  realidade  do  aluno,  acredita-se  que,  num  país  de  baixa  literatação, como é o nosso caso, uma disciplina com o grau de abstração e contextualização conceptual e histórica, como ocorre com a Filosofia, supõe que à opção de curso que for feita deve corresponder um cuidado redobrado com respeito às metodologias e materiais didáticos, levando sempre em conta as competências de que os alunos já dispõem e o que é necessário para  introduzi-los  significativamente  no  filosofar.” (Idem, p. 52)
	}, 
	entretanto, parecem existir diversas possibilidades didáticas 
	que levam em conta o processo de ensino da filosofia como a 
	transmissão de uma prática, mais do que um acúmulo informacional. 
	Os discutiremos ao longo desta seção. 

	A abordagem da CMSP parece, sim, apontar mais para este norte do que 
	aquela proposta pela SEED. Esta última talvez o incentive no contexto 
	das atividades. Mas, dentro do escopo das aulas remotas, 
	sua abordagem ainda é notavelmente mais 
	calcada no tradicional “conteudismo”. As questões feitas em aula têm 
	como claro objetivo mais a garantia da apreensão de uma determinada 
	ideia básica do que seu exame crítico. 

	Em ambos os casos, entretanto, limita-se a aula a um certo espaço 
	temporal limitado, condensando informações em tópicos curtos ou 
	discussões pontuais, procurando não extrapolar 
	o \textit{attention span} típico do estudante secundarista. Este 
	\textit{attention span} parece menor a cada dia e fica a impressão de que 
	no calor da batalha, sacrifica-se a riqueza de conteúdo e alguma 
	possibilidade de discussão pós-conteúdo em nome da transmissão de um 
	mínimo de ideias. 
	
	Há, principalmente na abordagem da SEED,
	um programa a ser seguido, e uma noção geral de que o conteúdo 
	de história da filosofia é parte de uma cultura geral que o aluno tem de 
	absorver, de alguma forma --- ``os conteúdos designam o conjunto de 
	saberes ou formas culturais cuja assimilação e apropriação pelos alunos e 
	alunas se considera essencial para seus desenvolvimentos e socialização.'' 
	(\footnote{\cite{obiols}, p. 68}). 
	Dentro de uma perspectiva mais ``tradicional'', 
	aprender a filosofar, paralelo ao desenvolvimento como cidadão e 
	indivíduo, depende de alguma cultura prévia. Para que a discussão 
	filósofica funcione, é preciso que haja um conteúdo a ser discutido. 
	
	Não é, entretanto, o entendimento de Obiols incompatível com os 
	preceitos dos parâmetros nacionais, uma vez que estes falam em uma 
	“apropriação dos textos” por parte dos estudantes
	\footnote{\cite{pcn}, p. 50} e estes textos, geralmente, terão, 
	por motivos culturais ou até mesmo religiosos uma relação 
	cultural pré-estabelecida com o estudante, ainda que de 
	maneira distante. 

	Por outro lado, este tipo de influência só pode ser medido ao longo de 
	alguns bons anos, é uma influência a longo prazo. Um dia, muito tempo 
	depois, você lembra de um nome que ouviu falar na escola, pesquisa mais a 
	respeito 	e este talvez seja o principal trunfo de uma 
	``cultura geral'': dar as referências para que, no futuro, você mesmo seja 
	capaz de encontrar o caminho das pedras, posteriormente. 

	A esta função, ambas as propostas parecem atender de maneira 
	satisfatória. Embora a pragmática proposta da SEED, fortemente 
	voltada ao vestibular, pareça faltar com a capacidade de 
	fornecer ao aluno um ponto de partida para, de fato, começar a 
	exercitar um pensamento crítico. É possível, contudo, que, no 
	contexto deste curso, tal tarefa fique a cargo não dos preceptores 
	das aulas online, mas de cada mestre, de cada turma. 

	Por certo, alguma informação estes alunos hão de reter, de todo o 
	modo. É isto, contudo, suficiente? 
	
	É difícil dizer. O aluno do qual se trata aqui é um adolescente, no final 
	de sua vida escolar. Idealmente, todos deveriam ter instrumental 
	suficiente para ``saber usar'' esta cultura geral que adquirem na escola. 
	A realidade, contudo, é que nunca adquirem esta maturidade a tempo. De 
	modo que tudo o que a escola repassa vira um grande bloco inexpugnável de 
	informações desconexas e relações confusas. Quando a um aluno as ideias 
	se dispõe desta maneira, até a noção de uma cultura geral está ameaçada. 
	
	Ou seja, é de se imaginar que não só deve se ensinar o aluno filosofia, 
	história e matemática, mas também deve ser feito algum esforço no sentido 
	de ensiná-lo a \textit{filosofar}. Obiols admite que dentro desta 
	perspectiva é possível acabar entrincheirado numa escola filosófica, posto 
	que a forma de filosofar é instrinsicamente ligada à escola filósofica à 
	qual se adere (\cite{obiols}, p. 70). Mas se o custo de dar ao aluno 
	instrumentos para pensar, inferir, conjecturar é assumir alguma escola 
	de pensamento, talvez não seja um mal negócio. 
	
	Ao que parece, a estrutura do curso da CMSP parece estar mais 
	próxima desta ideia, constituindo num interessante meio termo 
	entre a referência cultural -- o texto, o conteúdo -- e seu uso 
	“prático” em discussões. Estão ali dispostas as escolas de 
	pensamento, as mais diferentes ideias e, frequentemente, os 
	alunos são convidados a adotá-las, e, efetivamente, delas 
	se apropriar, como aqui falávamos. 

	%%%%%% >> Trecho excluído, por ora <<
	% O pensamento 
	% vira matéria morta. É como se outrora Descartes escrevesse em suas 
	% Meditações uma série de coisas, mas que é isto... São apenas coisas 
	% escritas --- que ficaram num distante passado. Ao expor o aluno ao 
	% pensamento de um autor, talvez não seja má ideia dar a corda, deixar 
	% que ele leia o livro, que se ``contamine'' pelas ideias do autor. 
	%%%%%% >> Fim do trecho excluído <<

	Ainda que não tenho o aluno o ímpeto de ler um livro --- é de se 
	esperar que, sobretudo em tempos de intenet e informações rápidas, 
	um adolescente não tenha a paciência de se dar ao árduo trabalho da 
	leitura ---, o professor ainda pode selecionar trechos, recortes 
	e deixar que o aluno veja razões para acreditar naquelas ideias. 
	Não há outra via de entrada para a filosofia. Não se a queremos viva. 
	
	É apenas um leve desvio em relação ao que os docente já fazem, e parece 
	se encaixar no projeto já existe (uma vez que os docentes já empregam  
	atividades com a turma que fogem à habitual aula expositiva, como já 
	fora mencionado aqui algumas tantas vezes) não seria de significativa 
	dificuldade. Isto, no entanto, requer abandonar um pouco a ``linearidade'' 
	do conteúdo e, em parte, o próprio conteúdo, se levarmos em conta o 
	reduzido tempo em sala de aula. 
	
	Não nos apeguemos, tanto ao currículo, porém. Ele é importante, como 
	já se viu pouco acima, mas de pouco adianta tê-lo sem ter os 
	instrumentos para manipulá-lo. Cabe ao professor dosar quanta ``praxis'' 
	colocará em suas aulas e quanto conteúdo por trás delas estará. 	
	Para A. Cerletti, a ideia de um currículo determinado, rígido, já é, 
	de saída, uma ``ficção'' (\cite{cerletti}, p. 79) e que o docente tem 
	papel ativo em formatar o currículo da maneira mais adequada: 
	
	\begin{citac}
		(...) nunca é possível ``aplicar'' ou pôr em execução um currículo 
		sem a intervenção criadora do docente, que é quem deverá assumir as 
		condições ``reais'' do ensino. No caso da filosofia, isso é 
		particularmente significativo. Se ensinar filosofia implica ensinar 
		a filosofar, deve-se esperar que sempre de quem ``aprende'' a 
		intervenção ativa no perguntar filosófico e na busca de respostas, e 
		isso não se pode levar adiante a não ser sob certas condições que o 
		professor deverá poder viabilizar. \cite{cerletti}, p. 79
	\end{citac}
	
	A sugestão do PCN quanto a este ponto sugere, sutilmente, um curso de 
	ação: 
	
	\begin{citac}
	(...) de um ponto de vista propedêutico, a conexão interna entre conteúdo e método deve tornar-se evidente: que o estudante tenha se apropriado significativamente de um determinado conteúdo filosófico significa, ao mesmo tempo, que ele se apropriou conscientemente de um método de acesso a esse conteúdo.  
	
		Apropriar-se do método adequado significa, primariamente, portanto, construir e exercitar a \textbf{capacidade de problematização}. Nisto consiste, talvez, a contribuição mais específica da Filosofia para a formação do aluno do Ensino Médio: auxiliá-lo a tornar temático o que está implícito   e   problematizar   o   que   parece   óbvio.   Portanto,   a   competência   de   \textbf{leitura significativa}   de   textos   filosóficos   consiste,   antes   de   mais   nada,   na   capacidade   de   problematizar o que é lido, isto é, \textbf{apropriar-se reflexivamente} do conteúdo. 
		\cite{pcn}, p. 50
	\end{citac} 
	
	Quando fala-se nesta tal de \textit{apropriação reflexiva}, já se 
	sugere, de certa forma, que uma possível maneira de apresentar o curso 
	seja tornar algumas ideias passíveis de apropriação por parte dos 
	estudantes. E, como o trecho ainda professa --- ``problematizar o que é 
	óbvio'' ---, ``confrontar'' as ideias trabalhadas com a ``realidade''. 
	
	Isto posto, é possível dizer que os cursos aqui comentados atendem 
	a esta demanda do ensino da filosofia como um “filosofar”? Penso ser 
	relativo. Como dito acima, depende um tanto da forma como estas 
	videoaulas/encontros virtuais se encaixam na infraestrutura do 
	ensino de filosofia, como um todo. 

	Se, em alguma turma, um professor propuser, por exemplo, uma redação, 
	em que o aluno seja incentivado a “filosofar” livremente, apenas se 
	armando de elementos ao seu redor, tem-se um prospecto de exercício 
	espiritual, por exemplo. 

	E, se, noutra turma, se propuser alguma atividade que exija do 
	aluno uma opinião acerca algum tópico discutido sobre as mais 
	diferentes escolas filosóficas (i.e. a existência ou não de Deus), 
	o arcabouço fornecido por um curso no estilo SEED pode muito bem 
	provar-se imensamente útil. 

	Retome-se o que diz Porchat, que apesar de crítico do estruturalismo 
	-- método que propugna o caráter integral da história da filosofia 
	ao ensino desta disciplina -- nos dias de hoje, não deixa de admitir 
	a basilar importância do estudo histórico da filosofia: 

	\begin{citac}
	É mais que evidente que não estou propondo que se minimize
	a importância de cursos historiográficos sobre o pensamento antigo,
	medieval e moderno. As disciplinas de História da Filosofia são o lugar
	natural desses cursos, seria apenas desejável que se privilegiassem os
	grande movimentos de pensamento que exerceram a influência que se
	sabe sobre toda nossa tradição filosófica ocidental. Um mínimo de
	iniciação se faz sempre necessária ao platonismo, ao aristotelismo, ao
	ceticismo, ao platonismo e aristotelismo medievais, ao cartesianismo,
	ao empirismo britânico, ao kantismo, ao hegelianismo. Se não a todas
	essas posturas filosóficas, pelo menos a uma parte considerável delas. 
		\footnote{\cite{porchat}, p. 25}
	\end{citac}

	Porchat, é claro, versa sobre o ensino superior de filosofia. 
	Contudo, prestar atenção à importância da historiografia não 
	machuca. Para que se filosofe, como já discutido aqui, é preciso 
	se ter \textit{algum} ponto de partida. No ensino superior, pode 
	ser a disciplina historiografica. Ao passo de que, no ensino 
	médio, pode ser o recorte, o texto. 

	Ainda neste mesmo artigo, Porchat defende que se permita ao aluno 
	“arriscar mais”. No ensino superior, como o próprio autor reconhece, 
	tal sugestão pode parecer polêmica. Mas, como parece demonstrar 
	a abordagem do CMSP, parece bastante adequado ao contexto do 
	ensino médio. 

	\newpage
	
	\section{Sobre a proposta didática}
	
	\subsection{Da aula remota}

	A “dinâmica” de uma aula remota é, em geral, radicalmente diferente 
	daquela observada em um encontro presencial, sem sombra de dúvida. 
	As limitações e possibilidades das aulas à distância são assuntos da 
	ordem do dia, em tempos em que, é verdade, não é possível fazer de 
	qualquer outra maneira.

	Se educação à distância não é novidade alguma -- já sendo, inclusive, 
	estopim de infindáveis debates entre educadores, desde que o avanço 
	das telecomunicações permitira -- \textit{em tese} -- que praticamente 
	qualquer um tivesse acesso 
	a este tipo de ensino --, é preciso lembrar que sua introdução ao 
	ambiente escolar tradicional tem sido apontada como potencialmente 
	disruptiva já há algum tempo. 

	Se, por um lado, se elimina o deletério efeito da algazarra 
	promovida pelos alunos ou o tempo gasto com organização da sala de 
	aula, por outro há uma seara de novos desafios a serem enfrentados 
	pelos docentes. No ambiente presencial é difícil controlar a 
	participação dos alunos, no ambiente virtual é, por outro lado, 
	garantir que ela exista de maneira efetiva. 

	Tem sido, em geral, uma insólita e inédita experiência a todos os 
	envolvidos e, indubtavelmente, gerará uma vasta gama de reflexões 
	importantes ao pensamento educacional de amanhã. Por ora, revela-se 
	difícil acompanhar de perto o aluno e, de súbito, avaliá-lo da forma 
	``tradicional'' se torna consideravelmente mais complexo. 

	Há, é claro, a possibilidade de se adotar uma avaliação formativa, 
	como proposto por Antoni Zabala. \footnote{cf. \cite{zabala}}
	Ainda assim, faz-se necessária uma 
	presença contínua do mestre. Uma “vantagem”, entretanto, deste 
	atual momento é que, por não ser responsável por ministrar a aula, 
	em si, o professor comumente estará mais disponível para sanar 
	dúvidas dos alunos via e-mail, efetivamente transformando-o n'algo
	mais próximo de um tutor. 

	\subsection{Das possíveis alternativas}

	Em contraposição a uma certa “rigidez” do conteúdo, talvez seja 
	conviente levantar algumas possibilidades iniciais de como se 
	trabalhar com a introdução do conceito filosófico. Principalmente 
	em turmas de marinheiros de primeira viagem, isto é, alunos que 
	nunca tiveram aulas de filosofia antes. 

	A este tipo de situação, a proposta de G. Arnaiz parece 
	particularmente adequada: trata-se das oficinas de filosofia, 
	apresentadas pelo próprio autor da seguinte maneira: 

	\begin{citac}
		Pois, sim, esta é a ideia. Converter a disciplina de 
		filosofia em uma prática, em que o mais importante seja 
		filosofar com outras pessoas, ao invés de ser obrigado 
		a escutar o insuportável monólogo do professor de 
		filosofia sobre uma série de autores e conceitos 
		filosóficos. 
		\footnote{\cite{arnaiz}, p. 2}
	\end{citac}

	Note-se, contudo, que, em razão do que fora discutido na seção 
	anterior, este tipo de organização tende a ser um pouco complexo, 
	no contexto da aula remota. Além disso -- principalmente no caso 
	da SEED -- as aulas parecem ser, como colocara algumas seções 
	atrás, mais \textit{parte do material} do que qualquer outra coisa. 
	Portanto, se há de existir algum tipo de oficina, provavelmente 
	terá de ocorrer entre o professor diretamente responsável pela 
	turma e os estudantes. 

	Arnaiz valoriza consideravelmente a questão e a discussão, mas 
	ressalta\footnote{cf. \cite{arnaiz}, p. 5} que, sobretudo, a 
	filosofia é, em si, uma espécie de exercício, muito semelhante 
	à meditação ou à yoga. O estímulo à discussão e à procura interna 
	por questões teria fundamental importância em formatar e fortalecer 
	o intelecto. 

	Desta maneira, o professor responsável talvez tivesse razoável 
	sucesso ao retonar às questões filosóficas básicas e procurar 
	conectá-las com o conteúdo que se observa nos vídeos. Ainda assim, 
	os vídeos da SEED têm um formato tão compacto e tão denso que, 
	possivelmente, seria necessário um imenso trabalho para reconstruir 
	as questões rumo aos conteúdos em si. 

	Os materias do CMSP, entretanto, parecem ter considerável êxito 
	em fazê-lo: a discussão é mais fluída, há participação interativa por 
	parte dos alunos (ainda que limitada) e tom mais informal dá a estes 
	encontro muito mais uma roupagem de discussão do que de conteúdo, 
	ajudando imensamente os alunos a fazer as conexões entre suas questões 
	e as clássicas discussões da filosofia. 

	Dentro do contexto da reflexão, é interessante notar que Arnaiz 
	não foge muito à ideia de um exercício espiritual, à moda de Hadot. 
	Entretanto, no caso de Arnaiz, este exercício parece ser 
	significativamente mais individual e instrospectivo. É, portanto, 
	mais delicado de se aplicar, se considerado o tamanho das turmas 
	com as quais normalmente tem se lidar. 

	Os exercícios espirituais, como discutido em Gallo, Genis, Kohan 
	e Wozniak tem muito mais a ver com uma apropriação 
	“filosófico-política” do meio e permite uma maior participação 
	coletiva. Tal \emph{approach} pode ser, aqui, de particular 
	interesse. Sobretudo se considerados, novamente, estes ditos 
	alunos de primeira viagem. De acordo com Kohan/Wozniak: 

	\begin{citac}
	O ato de ler o mundo como um texto é uma experiência singular de
pensamento, impossível de ser reproduzida, ainda mais quando praticada por um
grupo de adultos não alfabetizados. A natureza oral dos textos e das discussões
torna cada experiência com a Filosofi a uma nova experiência, porque o “texto”
muda conforme a experiência dos participantes se transforma. Conforme a leitura
do mundo se transforma, a leitura das “palavras” também se transforma.
	\cite{wozkoh}, p. 197
	\end{citac}

	Logo, a bem da verdade, a experiência por estes autores proposta 
	precede imensamente o texto. Trata-se de uma filosofia das questões 
	“naturais”. Dos cursos aqui analisados, observa-se que nenhum faz 
	mais do que tanger estas questões. No fim do dia, ambos seguem um 
	certo currículo. Quando se fala em “exercício espiritual”, ao fim 
	e ao cabo, este currículo acaba relegado a um distante segundo 
	momento. 

	Em minha proposta didática, sirvo-me desta ideia para o pontapé 
	inicial e, noutro momento, há o texto. Não o denso e dificultoso 
	texto, mas um texto clássico, ligado a não mais do que umas duas 
	ou três questões. A densidade observada nos encontros da SEED 
	absolutamente foge a esta medida. 

	Quanto ao método do CMSP, observa-se que existe uma ideia de 
	começar a aula por uma questão, procurando dar aos alunos uma 
	limitada oportunidade dar algum tom à aula. Porém, no fim do dia, 
	o que prevalece ainda continua sendo o conteúdo. 

	Escolher a justa medida entre o “conteúdo” e a discussão é, em 
	si, uma questão filosófica. Em nenhum outro conteúdo programático, 
	é tão importante a “intervenção criadora” do docente quanto em 
	filosofia: “não haveria então uma mandeira paradigmática [...] de 
	ensinar tal qual o tema da filosofia, já que o ensino filosófico 
	se constrói no diálogo filosófico do dia a dia“ 
	\footnote{\cite{cerletti}, p. 82}. 

	Em comum a todas estas \textit{approachs} citadas neste ensaio, 
	há, em comum, um foco na questão filosófica. No fim das contas, 
	sempre deve ser a questão o norte da elaboração de um curso ou 
	encontro. O conteúdo, como nos diz Porchat, é importante, mas, 
	como nos diz Cerletti: “Ensinar filosofia é dar lugar ao 
	pensamento do outro“\footnote{Idem, p. 87}. 

	Enfim, apesar de complexo no atual contexto, nunca se pode 
	perder de vista que a experiência da discussão e da troca de 
	experiências é míster à construção esforço filosófico comum. 
	Outras possíveis propostas sempre partiriam, idealmente, 
	deste ponto. E, neste aspecto, a proposta do CMSP parece estar 
	“no caminho correto.”

	\newpage
	
	\section{Epílogo}

	Não há como encerrar este ensaio sem ao menos um laivo de ceticismo, 
	contudo. Independentemente do meio, ou da mensagem, a delicadeza 
	devida às questões educacionais é, de toda a sorte, impossível na 
	contemporaneidade. A reificação do pensamento iluminista demarcara 
	certos imperativos que, para sempre, hão de estampar o que se 
	classifica ou não como “bom pensamento.” Será que para todo o sempre 
	há de se cair no dualismo entre educação elitista e educação 
	democrática quando se criticam ideias como o ensino em massa, 
	por exemplo? 

	Em geral, educar-se um único indivíduo já é uma tarefa difícil. 
	Não por menos, os antigos o faziam por etapas. E ao que hoje se 
	comumente chama “pensamento crítico” é um exercício ao qual o 
	jovem só podia se dispor após certa idade -- 21 anos, conforme 
	informado por Platão e Aristóteles. Além disso, por muitos 
	séculos, a sólida formação intelectual do Quadrivium seria 
	comprometida, faltassem as sólidas bases dadas pelo Trivium. 

	Logo, é de se imaginar que o sucesso em \textit{filosofar} 
	dependa, em certa escala, da pré-existência de uma articulação 
	lógica e linguística que não costuma estar disponível aos 
	estudantes que encontram, de súbito, a filosofia dentre as 
	matérias do ensino médio. 

	Certas propostas, como a dos chamados “exercícios espirituais”, 
	procuram mitigar esta lacuna à lá Paulo Freire: apropriando-se 
	de elementos do entorno. Se, por um lado, estimular uma percepção 
	do ambiente enquanto sobre ele se reflete não é de todo má ideia, 
	por outro não supre, de todo, o indivíduo de bases suficientes 
	para uma abstração mais ampla do mundo. 

	Um pouco mais distantes desta ideia estão pensadores como Cerletti e 
	Obiols, defendendo a abordagem da questão filosófica, desta vez 
	indo, ainda que de forma limitada, aos textos, evocando as clássicas 
	querelas filosóficas, retratadas das mais diferentes formas ao longo 
	de toda a história da filosofia. Esta saída parece apresentar 
	melhor balanço entre a discussão e as bases. 

	Ainda assim, faltará, 
	ao aluno médio, os elementos linguísticos para compreender certos 
	conceitos e intertextos presentes nas obras clássicas, o que fará 
	com que, com frequência, dependa do professor para debelar trechos 
	mais complexos. E o professor, em toda a sua ocupação, dificilmente 
	será capaz, sempre, de prover ao aluno satisfatória explicação 
	metodológica para que seja este aluno capaz de proceder por 
	conta própria. 

	Dentre os docentes de filosofia parece ser prática comum fornecer 
	nas aulas apenas pontos mais genéricos enquanto orientam mais 
	detalhadamente apenas aquele limitado montante de alunos que os 
	procuram nos intervalos de aula e no horário da saída, o que 
	parece evidenciar que, no limite, embora seja ideal que todos 
	tenham a \emph{informação}, apenas alguns de fato nela se 
	interessarão. E, em larga medida, estes precisarão se dar a um 
	esforço individual (e até autodidata) de aprendizado. 

	Ou seja, por mais amplos e interessantes que sejam os múltiplos 
	projetos didáticos, ao pensar na educação de maneira coletiva 
	sempre se corre o risco de deixar alguns para trás. Por outro 
	lado, teriam estes poucos sequer acesso à informação? Talvez 
	precisemos encontrar de se ensinar. É possível que o mofado 
	ambiente da sala de aula não sirva mais. 

	A frenética contemporaneidade não comporta mais o ar parado da 
	sala de aula. E a transmissão de informação tem sido excessiva e 
	de difícil lido ao estudando de ensino básico. Talvez precisemos 
	transformar o professor em uma espécie de tutor, capaz de guiar 
	o aluno por um método de pesquisa, ao mesmo tempo que se incentiva 
	o incentiva a proceder no estudo de forma individual, dando a ele 
	acesso a uma estação de trabalho e aos recursos multimídia devidos. 

	Mas são estas discussões para a posteridade. Por certo, não há 
	mais espaço racional para “privilegiar as cabeças cheias”. Porém, 
	por outro lado, urge repensar a forma como a discussão metodológica 
	e factual chega a cada estudante. 

	\newpage
	
	\bibliographystyle{abnt-alf}
	\bibliography{../bibliography}


\end{document}

